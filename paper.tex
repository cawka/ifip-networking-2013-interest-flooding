\documentclass[conference]{IEEEtran}
\title{Mitigating (D)DoS in Named-Data Networking}%:\\ Cache/Content Poisoning}
\author{anonymous}
\date{}
\begin{document}
\maketitle

\begin{abstract}
%
abstract
%
\end{abstract}

\noindent {\bf Keywords:} Future Internet Architectures;
Content-Centric Networks; Information-Centric Networks, Named-data
Networking; Security; Denial-of-Service; Distributed
Denial-of-Service.

\section{Introduction \label{intro}}
Current Internet architecture and its resilience to DoS and DDOS. Requirements from a new architecture such as CCN with respect to DoS and DDoS resilience.

Our contributions.

Paper organization.

\section{ CCN Overview\label{ccn-intro}}

\section { DDoS Attacks in CCN \label{ccn-ddos}}
Brief overview of possible attacks in CCN, and scope of what we will tackle in this paper (Interest flooding).

\section { Interest Flooding in CCN }

\section {Countermeasures against Interest flooding}

Basic limits, probablity based on statistics, and dynamic window adjustment

(Description of each of the above, and their implications) 


\section {Evaluation}
 Both simulation, as well as emulation for various sized topologies (trees as well as real topologies), various parameters etc.
List all possible parameters, say clearly which ones we vary, and which ones we do not, along with explanations.

Metrics that we will consider in our evaluation (Satisfaction rate for good clients, Link utilization near producers, Latency for good clients, good versus bad interests as a function of time).

\subsection{Simulation versus Emulation}
Results establishing validity of  simulator.

\subsection{Results from probability based on statistics}

\subsection{Results from dynamic window adjustment}

\section{Implications/Analysis}
Lessons we learnt, comparison of different mitigation techniques. 
Adapting these to mitigate other DDoS attacks.

\section{Conclusions and Future Work}


\bibliographystyle{plain}
\bibliography{references}


\end{document}
