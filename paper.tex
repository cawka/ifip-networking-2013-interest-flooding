\documentclass[conference]{IEEEtran}
\title{Mitigating (D)DoS in Named-Data Networking}%:\\ Cache/Content Poisoning}
\author{anonymous}

\usepackage{graphicx}
% \usepackage[colorlinks]{hyperref}
\usepackage[]{hyperref}
\usepackage{breakurl}
\usepackage{url}
\usepackage[nocompress]{cite}
% \usepackage{verbatim}
% \usepackage{algpseudocode}
\usepackage{algorithm}
\usepackage{algpseudocode}

\begin{document}
\maketitle

\begin{abstract}
%
abstract
%
\end{abstract}

\noindent {\bf Keywords:} Future Internet Architectures;
Content-Centric Networks; Information-Centric Networks, Named-data
Networking; Security; Denial-of-Service; Distributed
Denial-of-Service.

\section{Introduction \label{intro}}
Current Internet architecture and its resilience to DoS and DDOS. Requirements from a new architecture such as CCN with respect to DoS and DDoS resilience.

Our contributions.

Paper organization.

\section{ CCN Overview\label{ccn-intro}}

\section { DDoS Attacks in CCN \label{ccn-ddos}}
Brief overview of possible attacks in CCN, and scope of what we will tackle in this paper (Interest flooding).

\section { Interest Flooding in CCN }

\section {Countermeasures against Interest flooding}

Basic limits, probablity based on statistics, and dynamic window adjustment

(Description of each of the above, and their implications) 


\section {Evaluation}
 Both simulation, as well as emulation for various sized topologies (trees as well as real topologies), various parameters etc.
List all possible parameters, say clearly which ones we vary, and which ones we do not, along with explanations.

Metrics that we will consider in our evaluation (Satisfaction rate for good clients, Link utilization near producers, Latency for good clients, good versus bad interests as a function of time).

\subsection{Simulation versus Emulation}
Results establishing validity of  simulator.

\subsection{Results from probability based on statistics}

\subsection{Results from dynamic window adjustment}

\section{Implications/Analysis}
Lessons we learnt, comparison of different mitigation techniques. 
Adapting these to mitigate other DDoS attacks.

\section{Conclusions and Future Work}

\bibliographystyle{plain}
\bibliography{references}

\appendix

\subsection{Physical limits}
\label{sec:physical limits}

The requirement to send an Interest in order to receive Data packet, provides an NDN consumer a unique opportunity to request the right amount of Data.
Moreover, the same opportunity to control the amount of data flow is given not only to consumers, but all routers between consumer and producer (or nearby caches).  
In other words, every node, either a consumer or an intermediate router, is able to control how much data it wants to receive by limiting the number of forwarded Interests.

The limitation can implemented in a number of different ways, including leaky bucket scheduling and window-based flow control.
We decided to following TCP-like window-based flow control and applied the sliding window approach to implement Interest limits.

The size of the window defines how many Interests can be send out before Interests get satisfied or expired.
From the one hand, this size should be large enough to ``fill the pipe,'' meaning that a node needs to send enough Interests to receive Data at full capacity of the incoming link.
On the other hand, the window's size should not be too large to avoid excessive buffering and congestion of the Data packet.
Thus, the ideal size for such a window need to be defined proportional to link's bandwidth-delay product~\cite{tcp-survey}.
With the objective to request as many Data packets, as downstream link can pump through, we are getting the following equation for Interest limit:

\begin{equation}
\mathrm{Interest\ Limit} = Delay\ [s] \cdot \frac{\mathrm{Bandwidth\ [Bytes/s]}}{\mathrm{Data\ packet\ size\ [Bytes]}}
\end{equation}

Note that the value of \textit{Delay} is not known a priory and varies between different Interest-Data flows.
However, we do not need to know the exact value of the delay and can set it as an average round trip delay among all flows (with a reasonable filtering of outliers).
This way, the statistical traffic multiplexing with link-level buffering will allow full utilization of the downstream link.
Exactly the same reasoning can be applied to the \textit{Data packet size} parameter, which can also be set to an average observed Data packet size.

Unlike rate-based approaches, window-based limiting does not require precise knowledge about the rate, as well does not need precise scheduling mechanisms.
Like in TCP, the window-based flow is self-clocking, easily adjusting itself to any traffic patterns.

\begin{algorithm}[h]
\caption{Simple physical limits}
\label{alg:simple limits}
\begin{algorithmic}[1]
\For{\textbf{each} interface \em{f}}
    \State $L_{f} \leftarrow$ Interest Limit according to (1)
    \State $O_{f} \leftarrow 0$ \Comment{Outstanding Interests on interface \textbf{f}}
\EndFor

\vspace{0.2cm}
\Function{InInterest}{Interest $i$, InInterface $if$}
\For{\textbf{each} OutInterface $of$ \textbf{in} FwDecision($i,if$)}
    \If{$O_{of} < L_{of}$} \Comment{\textbf{of} is under physical limits}
        \State $O_{of} \leftarrow O_{of} + 1$
        \State Forward($i$, $of$)
    \Else
        \State drop Interest
    \EndIf
\EndFor
\EndFunction

\vspace{0.2cm}

\end{algorithmic}
\end{algorithm}


%%% Local Variables: 
%%% mode: latex
%%% TeX-master: "../paper"
%%% End: 


% \subsection{Physical limits with ``fair'' queuing}
% \label{sec:queuing}

% However, this limitation approach does not attempt to utilize data plane performance knowledge (i.e., Interest satisfaction ratio statistics) to discriminate good and malicious Interests.

To partially overcome deficiencies of the Physical limits algorithm we need to ensure that the forwarded Interests represent at least a ``fair'' mix of the Interests received from different neighbors (interfaces).
That is, if the routers A on Fig.~\ref{fig:flooding example} has a very tight token budget, these tokens should be fairly distributed between incoming interfaces \texttt{eth0} and \texttt{eth1}.
Because of the very small volume of Interests, we cannot simply rely on network buffers to do statistical multiplexing of Interests, as they would almost never be buffered.
At the same time, until bag of tokens is not empty, there is no reason to delay Interest forwarding, as we do not known how many and from which interfaces Interests will arrive in the future.
Therefore, in order to achieve the goal of ``fair'' mixing of Interests, we need to implement additional mechanisms to buffer and mix incoming Interests, only if they cannot be immediately forwarded.

For the buffering part, we can reuse Pending Interest Table, with a small extension to support flagging of the Interests that cannot be forwarded immediately (see example on Fig.~\ref{fig:queueing}). 
As for the mixing part, we need an additional fair queuing mechanism, which can be implemented in a form of hierarchical queues (on Fig.~\ref{fig:queueing})\footnote{This essentially is a class based queuing, with classes for each outgoing/incoming interface.} or using virtual time approach~\cite{zhang1990virtual}. 
It should be noted that unlike normal queuing, Interest queues do not actually store a packet, but merely a bi-directional pointer to the existing PIT entry.
This way, PIT entry can be quickly updated when the Interest is actually forwarded, as well as the element can be removed from the queue when the Interest expires.

\begin{figure}[htbp]
  \centering
  \includegraphics[scale=0.65]{queue}
  \caption{Interest queuing: if tokens are unavailable, the router creates PIT entry, but instead of forwarding, enqueued the Interest}
  \label{fig:queueing}
\end{figure}

A more formalized description of the Physical limits algorithms with per-interface fairness is presented in Pseudocode~\ref{alg:queuing}.
The algorithm extends the base Physical limits algorithm by enabling queuing when the bag of tokens is empty (lines 7--10), as well as by triggering an action (lines 16--21), when a token becomes available and enqueued Interest can be finally forwarded.
At the same time, the algorithm limits number of Interests allowed in a queue, constraining memory usage increase by at most a constant factor, compared to the base Physical limits algorithm (i.e., memory attack on routers are still unfeasible).


\floatname{algorithm}{Pseudocode}

%%%%%%%%%%%%%%%%%%%%%%%%%%%%%%
%%%%%%%%%%%%%%%%%%%%%%%%%%%%%%
%%%%%%%%%%%%%%%%%%%%%%%%%%%%%%

\begin{algorithm}[h]
\caption{Physical limits with per-interface fairness}
\label{alg:queuing}
\begin{algorithmic}[1]
\State{} \Comment{Same initialization, InData and Timeout functions as in Physical Limits algorithm}

\vspace{0.2cm}

\Function{OutInterest}{Interest \textbf{i}, InInterface \textbf{if}, OutInterface \textbf{of}}
    \If{$L_{of} - O_{of} > 0$} \Comment{\textbf{of} is under physical limits}
        \State $O_{of} \leftarrow O_{of} + 1$  \Comment{``Borrow'' token}
        \State add \textbf{of} to PIT entry and forward \textbf{i} to \textbf{of}
    \Else
        \State Queue $q \leftarrow of$.GetSubQueue($if$)
        \If{$Size(q) < L_{of}$}
           \State $q$.PushInterest($i$)
           \State add \textbf{of} to PIT entry, and link PIT entry with the queue
        \Else
           \State drop Interest
        \EndIf
    \EndIf
\EndFunction

\vspace{0.2cm}
\State{} \Comment{\textit{Whenever $L_{of} - O_{of}$ becomes larger than zero}}
\Function{TokenBecomesAvailable}{}
    \State Queue $q \leftarrow$ $of$.GetRoundRobinSubQueue 
    \State $i \leftarrow$ $q$.PopInterest
    \State update PIT entry and Forward($i$, $of$)
\EndFunction
\end{algorithmic}
\end{algorithm}


It should be noted that enqueued Interests should not be kept in the queue for a prolonged period of time.
Otherwise, by the time the Interests reaches the Data, the state could have been long expired downstream, effectively making such an Interest useless.
Additional mechanisms of pair-wise agreements between NDN routers and periodic Interest refresh can solve this particular problem, but it is out of the scope of the present paper.

As we show in Section~\ref{sec:evaluation}, fair queueing provides a partial relief from the Interest flooding attack, allowing legitimate users to successfully fetch Data for 15--20\% of the expressed Interests (compared to 0-10\% without fair queueing).
At the same time, the Physical limits with or without fair queueing allows attackers to send a relatively small volume of Interests in order to significantly impact service for the legitimate users.
Therefore, to successfully solve the problem, we need a fundamentally different, more intelligent approach, allowing localization of the attack traffic as close as possible to the attack origin.

%%% Local Variables: 
%%% mode: latex
%%% TeX-master: "../paper"
%%% End: 


% \subsection{Dynamic limits}

\subsection{Dynamic limits assignment algorithm}
\label{sec:dynamic limits}

A notable problem of the queuing method is that the method tries to differentiate between good and malicious Interests, but does not effectively limit the amount of malicious Interests.
That is, if an attacker has a large number of malicious Interests to send, it will receive a slightly lower priority in sending these Interests (some will be delayed more, some number will be dropped), but a lot of these Interests will get through.

One way to solve this problem is announce and dynamically readjust Interest limits for classes (e.g., per prefix and per incoming face), based on Interest satisfaction ratios for those classes. 
Pseudocode~\ref{alg:dynamic limits} summarizes our proposal, which is a variation of a well-known push-back mechanism.


\floatname{algorithm}{Pseudocode}

%%%%%%%%%%%%%%%%%%%%%%%%%%%%%%
%%%%%%%%%%%%%%%%%%%%%%%%%%%%%%
%%%%%%%%%%%%%%%%%%%%%%%%%%%%%%

\begin{algorithm}[h]
\caption{Dynamic limits}
\label{alg:dynamic limits}
\begin{algorithmic}[1]

\Function{AnnounceLimits}{} \Comment{\textit{E.g., every second}}
\For{\textbf{each} prefix $p$ in FIB}
    % \State{} \Comment{Get total limit for the prefix $p$}
    \State $L \leftarrow \displaystyle\sum\limits_{\mathrm{\forall \mathit{of} \in FIB(\mathit{p})}}{}{}of$.GetPerPrefixLimit($p$)
    \State $R \leftarrow 0$
    \For{\textbf{each} available face $f$}
        \State $l_f \leftarrow L \times p$.$f$.GetSatisfactionRatio()
        \State $R \leftarrow R + p$.$f$.GetSatisfactionRatio()
    \EndFor

    \If{$\displaystyle\sum\limits_{\forall f}^{}l_f < l$} \Comment{If under-utilization detected}
        \State $\forall f : l_f \leftarrow \displaystyle\frac{l_f}{R}$ \Comment{Normalize limits}
    \EndIf

        \State AnnounceLimit($f$, $p$, $l$)
\EndFor
\EndFunction

\vspace{0.2cm}

\State{} \Comment{\textit{When announcement from the neighbor is received}}
\Function{InLimits}{InFace $if$, Prefix $p$, Limit $l$}
    \State $if$.GetPerPrefixLimit($p$) $\leftarrow l$ 
\EndFunction

\end{algorithmic}
\end{algorithm}

When an Interest arrives, in addition to be checked against per-outgoing interface limits, it is subjected to per-prefix per-incoming interface limits (lines 3 and 4 in Pseudocode~\ref{alg:dynamic limits}).
In other words, for each prefix in FIB a node cannot send more Interests than a specified limit for this prefix (``dynamic limits''), as well as the node cannot send out more Interests out of the face, than the limit specified for this face (``physical limit'').
While the latter is defined in the same manner as in Section~\ref{sec:physical limits}, ``dynamic limits'' are periodically announced to all neighbors (\textit{AnnounceLimits} function) and adjusted based on received announcements from neighbors (\textit{InLimits} function).%
\footnote{The initial value (not specified in the pseudocode) for the dynamic limits is the corresponding physical limits.}

During the announcement phase, a node tries to determine ``quality'' of neighbors in terms of Interest satisfaction rations and tries to give more resources (accept more Interests from) neighbors with better satisfaction ratios.
At the same time, it is undesirable to over-limit neighbors when there are available resources.
That is, when the sum of all limits calculated by directly applying the satisfaction ratios to the total available limit (lines 15-18 in Pseudocode~\ref{alg:dynamic limits}), limits should be normalized so the sum of all announced limits is at least equal to the total available limit (lines 19-21).

When the satisfaction ratio for a particular prefix and incoming interface becomes close to zero, a node starts announcing (and in real implementation, enforcing) the zero limit.
The node will keep announcing this zero limit for a time period, depending on statistics degradation curve (see more in Section~\ref{sec:stats}).

The dynamic limits method can work as standalone algorithm, as well as it can be (and we believe that it should be, and all results presented in this paper are based on) coupled with the queuing algorithms.
This way, the dynamic limits part pushes back malicious traffic, while queueing part provides limited per-incoming interface fairness for good traffic.


%%% Local Variables: 
%%% mode: latex
%%% TeX-master: "../paper"
%%% End: 



Apart from the Interest satisfaction statistics generation, there is a question how this statistics can be used to actually enforce prioritization and penalizing of Interests.
A straightforward way to achieve this enforcement is to consider Interest satisfaction rate as a direct probability to accept (forward) or reject an incoming Interest (see Pseudocode~\ref{alg:probabilistic model}).

\floatname{algorithm}{Pseudocode}

%%%%%%%%%%%%%%%%%%%%%%%%%%%%%%
%%%%%%%%%%%%%%%%%%%%%%%%%%%%%%
%%%%%%%%%%%%%%%%%%%%%%%%%%%%%%

\begin{algorithm}[h]
\caption{Probabilistic model}
\label{alg:probabilistic model}
\begin{algorithmic}[1]
\State{} \Comment{Same initialization, InData and Timeout functions as in Physical Limits algorithm}

\vspace{0.2cm}
\Function{OutInterest}{Interest \textbf{i}, InInterface \textbf{if}, OutInterface \textbf{of}}

    \State{} \Comment{Use uniform probability distribution model $P(X)$}
    \State{} \Comment{$P(X) : \forall x \in [0,1] \Rightarrow P(x) = x$}
    
    \If{$F_{if} > \theta $} \Comment{At least some Interests were forwarded before}
        \State $s \leftarrow (1 - U_{if} / F_{if})$
        \State Drop interest with probability $P(s)$
    \EndIf

    \State{forward the Interest, subjecting to physical limits}
\EndFunction

\end{algorithmic}
\end{algorithm}

Parameter $\theta$ on line 5 of the Pseudocode~\ref{alg:probabilistic model} ensures that the probabilistic model is not applied when there a very small volume of Interests coming from the particular interface.
That is, while we want to deny service to the attackers, we also want give an opportunity for the users who do not abuse the network to regain their share of resources after temporary Data delivery failures.

A main drawback of the probabilistic Interests accept method is the fact that each router on the path make an independent decision on what to do with the Interest.
As a result of such independent decisions, the probability of legitimate Interests not to be dropped decreases quickly as the number of hops between the requester and Data grows, worsening the satisfaction statistics and resulting in further drops.
In example on Fig.~\ref{fig:flooding example}, the router A observes 50\% satisfaction rate for \texttt{eth1} and 0\% rate for \texttt{eth0}. 
At the same time, the router B sees the satisfaction rate for its \texttt{eth0} interface at 30\% level.
Next time a legitimate Interests arrives on the router A, it has 50\% chance to be forwarded further, and if forwarded, it has only $50\% \times 30\% = 15\%$ chance to be actually send towards the Data producer.
To prevent such overreacting, we need to enable some form of a gossiping protocol between neighboring NDN routers, in order to explicitly specify amount of Interests that each router is willing to forward.

%%% Local Variables: 
%%% mode: latex
%%% TeX-master: "../paper"
%%% End: 


\subsection{Satisfaction rate statistics generation algorithm}

Statistic generation is performed at different time and name prefix granularities.  
Figure~\ref{fig:stats tree} shows an overall design of the implemented statistics generation module and Pseudocodes~\ref{algo:loadstats}, \ref{algo:interest stats}, \ref{algo:prefix stats}, \ref{algo:stats} show implementation details.

\begin{figure}[htpb]
  \centering
  \includegraphics[scale=0.7]{figures/stats-illustration.pdf}
  \caption{Statistics tree}
  \label{fig:stats tree}
\end{figure}

A separate set of statistical information is kept for each Interest name, as well as aggregated to all Interest name prefixes, including root (``/'') prefix (per-prefix statistics tree on Figure~\ref{fig:stats tree}).

\floatname{algorithm}{Pseudocode}

%%%%%%%%%%%%%%%%%%%%%%%%%%%%%%
%%%%%%%%%%%%%%%%%%%%%%%%%%%%%%
%%%%%%%%%%%%%%%%%%%%%%%%%%%%%%

\begin{algorithm}[h]
\caption{Stats component}
\label{algo:loadstats}
\begin{algorithmic}[1]
\State $\alpha_1 \leftarrow e^{-1.0/5.0}$  \Comment{$\approx$ 5~sec average}
\State $\alpha_2 \leftarrow e^{-1.0/30.0}$ \Comment{$\approx$ 30~sec average}
\State $\alpha_3 \leftarrow e^{-1.0/60.0}$ \Comment{$\approx$ 60~sec average}
\State $avg_1 \leftarrow 0$ \, $avg_2 \leftarrow 0$ \, $avg_3 \leftarrow 0$ \, $counter \leftarrow 0$

\vspace{0.2cm}
\Function{ProcessEvent}{new Event}
\State $counter \leftarrow counter + 1$
\EndFunction

\vspace{0.2cm}
\Function{Advance}{} \Comment{Every second}
\State {} \Comment{\textit{Exponential weighted moving average smoothing}}
\State $avg_1 \leftarrow \alpha_1 \cdot avg_1 + (1 - \alpha_1) \cdot counter$ 
\State $avg_2 \leftarrow \alpha_2 \cdot avg_2 + (1 - \alpha_2) \cdot counter$
\State $avg_3 \leftarrow \alpha_3 \cdot avg_3 + (1 - \alpha_3) \cdot counter$
\State $counter \leftarrow 0$
\EndFunction
\end{algorithmic}
\end{algorithm}

%%%%%%%%%%%%%%%%%%%%%%%%%%%%%%
%%%%%%%%%%%%%%%%%%%%%%%%%%%%%%
%%%%%%%%%%%%%%%%%%%%%%%%%%%%%%

\begin{algorithm}[h]
\caption{Interest Stats component}
\label{algo:interest stats}
\begin{algorithmic}[1]
\State $I \leftarrow$ new Stats   \Comment{\# of Interests}
\State $S \leftarrow$ empty Stats \Comment{\# of satisfied Interests}
\State $U \leftarrow$ empty Stats \Comment{\# of unsatisfied Interests}

% \vspace{0.2cm}
% \Function{ProcessEvent}{new Event}
% \State $I \leftarrow counter + 1$
% \EndFunction

\vspace{0.2cm}
\Function{Combine}{other}
\State $I.counter \leftarrow other.I.counter $ 
\State $S.counter \leftarrow other.S.counter $
\State $U.counter \leftarrow other.U.counter $
\EndFunction

\vspace{0.2cm}
\Function{GetSatisfiedRatio}{}
  \For{each granularity $i$}
     \If{$I.avg_i < 0.1$}
        \State \Return $-1$ \Comment{Unknown state}
     \Else
        \State \Return $\frac{S.avg_i}{I.avg_i}$ \Comment{Interest satisfaction ratio}
     \EndIf
  \EndFor
\EndFunction

\vspace{0.2cm}
\Function{Advance}{} \Comment{Every second}
\State $I.$advance()
\State $S.$advance()
\State $U.$advance()
\EndFunction

\end{algorithmic}
\end{algorithm}

%%%%%%%%%%%%%%%%%%%%%%%%%%%%%%
%%%%%%%%%%%%%%%%%%%%%%%%%%%%%%
%%%%%%%%%%%%%%%%%%%%%%%%%%%%%%

\begin{algorithm}[h]
\caption{Prefix stats component}
\label{algo:prefix stats}
\begin{algorithmic}[1]
\State $T \leftarrow$ empty Interest Stats \Comment{Total per-prefix stats}
\For{\textbf{each} Face f}
    \State $PI[f] \leftarrow$ empty Interest Stats \Comment{per-in Face}
    \State $PO[f] \leftarrow$ empty Interest Stats \Comment{per-out Face}
\EndFor


\vspace{0.2cm}
\Function{NewPitEntry}{}
  \State $T.I.counter \leftarrow T.I.counter + 1$ %\Comment{Total \# of Interets}
\EndFunction

\vspace{0.2cm}
\Function{Incoming}{Face}
  % \State {} \Comment{per-in Face \# of Interests}
  \State $PI[Face].I.counter \leftarrow PI[Face].I.counter + 1$
\EndFunction

\vspace{0.2cm}
\Function{Outgoing}{Face}
  % \State {} \Comment{per-out Face \# of Interests}
  \State $PO[Face].I.counter \leftarrow PO[Face].I.counter + 1$
\EndFunction

\vspace{0.2cm}
\Function{Satisfy}{}
  \State $T.S.counter \leftarrow T.S.counter + 1$ %\Comment{Total \# of Interets}

  \For{\textbf{each} Face $f$}
    % \State {} \Comment{per-in Face \# of Interests}
    \State $PI[f].S.counter \leftarrow PI[f].S.counter + 1$
    % \State {} \Comment{per-out Face \# of Interests}
    \State $PO[f].S.counter \leftarrow PO[f].S.counter + 1$
  \EndFor
\EndFunction

\vspace{0.2cm}
\Function{Timeout}{}
    \State $T.U.counter \leftarrow T.U.counter + 1$ %\Comment{Total \# of Interets}

    \For{\textbf{each} Face $f$}
        % \State {} \Comment{per-in Face \# of Interests}
        \State $PI[f].U.counter \leftarrow PI[f].U.counter + 1$
        % \State {} \Comment{per-out Face \# of Interests}
        \State $PO[f].U.counter \leftarrow PO[f].U.counter + 1$
    \EndFor
\EndFunction

\vspace{0.2cm}
\Function{Combine}{other}
    \State $T.$Combine($other.T$)
    \For{\textbf{each} Face $f$}
        \State $PI[f].$Combine($other.PI[f]$)
        \State $PO[f].$Combine($other.PO[f]$)
    \EndFor
\EndFunction

\vspace{0.2cm}
\Function{Advance}{} \Comment{Every second}
\State $T.$advance()
\For{\textbf{each} Face $f$}
    \State $PI[f].$advance()
    \State $PO[f].$advance()
\EndFor

\EndFunction

\end{algorithmic}
\end{algorithm}

%%%%%%%%%%%%%%%%%%%%%%%%%%%%%%
%%%%%%%%%%%%%%%%%%%%%%%%%%%%%%
%%%%%%%%%%%%%%%%%%%%%%%%%%%%%%

\begin{algorithm}[h]
\caption{Statistics generation algorithm}
\label{algo:stats}
\begin{algorithmic}[1]

\State stats $\leftarrow$ empty tree of Prefix stats \Comment{Stats tree}

\vspace{0.2cm}
\Function{ProcessEvent}{new Interest, inFace}
  \State stats[Interest.Name].NewPitEntry()
  \State stats[Interest.Name].Incoming(inFace);
  \State stats[Interest.Name].Timeout()
\EndFunction

\vspace{0.2cm}
\Function{ProcessEvent}{Interest rejected, inFace}
  \State stats[Interest.Name].NewPitEntry()
  \State stats[Interest.Name].Incoming(inFace)
\EndFunction

\vspace{0.2cm}
\Function{ProcessEvent}{Interest forwarded, outFace}
  \State stats[Interest.Name].Outgoing(outFace)
\EndFunction

\vspace{0.2cm}
\Function{ProcessEvent}{pending Interest timeout}
  \State stats[Interest.Name].Timeout()
\EndFunction

\vspace{0.2cm}
\Function{ProcessEvent}{pending Interest satisfied}
  \State stats[Interest.Name].Satisfy()
\EndFunction

\vspace{0.2cm}
\Function{Periodic}{every second}
\For{every node, walking from children to root}
    \State $node.parent.$Combine($node$)
    \State $node.$Advance()
\EndFor
\EndFunction


\end{algorithmic}
\end{algorithm}

%%% Local Variables: 
%%% mode: latex
%%% TeX-master: "../paper"
%%% End: 





\end{document}
