% \documentclass[10pt, conference, compsocconf]{IEEEtran}
\documentclass[conference]{IEEEtran}
% http://networking2013.poly.edu/call-for-papers/paper-submission/
% Page limit 9 pages

\makeatletter
\def\ps@headings{%
\def\@oddhead{\mbox{}\scriptsize\rightmark \hfil \thepage}%
\def\@evenhead{\scriptsize\thepage \hfil \leftmark\mbox{}}%
\def\@oddfoot{}%
\def\@evenfoot{}}
\makeatother

\pagestyle{headings}


\title{Interest Flooding Attacks and Countermeasures in Named Data Networking}%:\\ Cache/Content Poisoning}

%I don't think submission is anonymous...
% I PUT AUTHOR NAMES IN ALPHEBETICAL ORDER. It is the common practice in security and I always list authors alphebetically in my publications. -Ersin 
\author{
    \IEEEauthorblockN{Alexander Afanasyev\IEEEauthorrefmark{1}, Priya Mahadevan\IEEEauthorrefmark{2}, Ilya Moiseenko\IEEEauthorrefmark{1}, Ersin Uzun\IEEEauthorrefmark{2}, Lixia Zhang\IEEEauthorrefmark{1}}
    \IEEEauthorblockA{\IEEEauthorrefmark{1}University of California, Los Angeles
    \\\{alex, ilya, lixia\}@ucla.edu}
    \IEEEauthorblockA{\IEEEauthorrefmark{2}Palo Alto Research Center
    \\\{ersin.uzun, priya.mahadevan\}@parc.com}
}

\newfont{\nicettfont}{cmtt10}
\newcommand{\ndnName}[1]{``{\nicettfont #1}''}
\renewcommand{\texttt}[1]{{\nicettfont #1}}

\newcommand{\todo}[1]{\vspace{2 mm}\par \noindent \marginpar{\textsc{ToDo}}
\framebox{\begin{minipage}[c]{0.95 \columnwidth}
\tt #1 \end{minipage}}\vspace{5 mm}\par}

\usepackage{graphicx}
% \usepackage[colorlinks]{hyperref}
\usepackage[]{hyperref}
\usepackage{breakurl}
\usepackage{url}
\usepackage[nocompress]{cite}
% \usepackage{verbatim}
% \usepackage{algpseudocode}
\usepackage{algpseudocode,algorithm}
% More customizeable version. Probably it would be better to convert pseudocode to 2e format
% \usepackage{algorithm2e} 
\usepackage{multirow}
\usepackage{times}
\usepackage{color}

\graphicspath{{figures/}}

\begin{document}

\maketitle

\begin{abstract}
Distributed Denial of Service attack is an ongoing problem in today's Internet, where packets from large numbers of compromised hosts thwart the paths to the victim site and/or overload the victim machines. 
In a newly proposed future Internet architecture, Named Data Networking (NDN), end users request desired data by sending Interest packets, and the network forwards Data packets upon request only, effectively eliminating many existing DDoS attacks. 
However, an NDN network can be subject to a new type of DDoS attack, namely Interest packet flooding.  
In this paper we investigate effective solutions to mitigate Interest flooding.
We show that NDN's inherent properties to keep per packet state on each router and flow balance (i.e., one Interest packet retrieves at most one Data packet) provides the  basis for effective DDoS mitigation algorithms.
Our evaluation through simulations shows that the solution can quickly and effectively respond and mitigate Interest flooding.
\end{abstract}

\begin{IEEEkeywords}
Information-centric networks, named-data networking, denial-of-service
\end{IEEEkeywords}

\section{Introduction}
\label{sec:intro}

% Introduction / Motivation
% Brief introduction about attacks in NDN
% Contributions of the paper

% Introduction to the problem: ndn, solves many problems, but has potential to introduce new problems.
% in this paer we are studing a number of remedies to Interest floodig attack, as well as trying to doscover ptential of tjese dolutions and their shortcomings

% Structure. Hopefully around three paragraps. 
% - what is happening with ndn and what is interest floodog
% - what are the general approaches we are taking, ranging from naïve to more intelligent
% - basic summary of the results

% move disclaimer from section 3 to the intro.

Various forms of distributed denial of services attack (DDoS) pose a significant and constant threat to the existing Internet infrastructure~\cite{arbor-report}.
The recently proposed Named Data Networking (NDN) architecture~\cite{ndn-conext, ndn-tr} completely changes the communication paradigm in the network by removing the notion of network host identities and making the application-level Data names the first-class network citizens.
That is, instead of requiring users to explicitly connect to specific servers and then use application-level protocols to request desired Data, users can directly ask the network for the Data.
% In other words, NDN changes communication from host-centric to data-centric.
Such a shift automatically solves several long standing problems, including source address spoofing and reflector DDoS attacks~\cite{mirkovic2004taxonomy}.

However, as long as there is still a notion of destination, requests (Interests in NDN terms) can still be send towards the destinations,%
\footnote{Note that ```destination'' is much more vague in NDN than in IP. 
Copies of the requested Data, towards which Interests may be forwarded, can be in many places, including long-term and short-term caches on NDN routers and alternative locations of the Data producer.} 
potentially disrupting service to legitimate users.
Taking into account that NDN requires the stateful forwarding of Interets---as the only way for routers to return Data back is to remember from where the Interest came in---an attacker may specifically target the stateful forwarding of NDN, e.g., in a form of Interest flooding attack. 

Our primary goal in this paper is not to completely solve all denial of service problems in NDN, but to demonstrate using the Interest flooding attack example (Section~\ref{sec:interest flooding}) that although new NDN-specific DDoS attacks are possible in NDN network, the network is fully equipped to mitigate effects of these attacks.
In the paper we design several algorithms to mitigate the Interest flooding attack (Section~\ref{sec:design}), exploring the following two unique venues provided by NDN.
First, an NDN router does not need to accept, create state, and forward all the incoming Interests, but only the amount that would fully utilize downstream link capacity.
Second, the fact that Data always follows the Interest paths can be used to distinguish between good (e.g., a forwarded Interest gets satisfied) and ``bad'' Interests. 

% Alex: Not sure whether it should be mentioned in intro or not.  If it should, then I don't really like it...
To quantify the quality of the designed attack mitigation algorithms, we performed a simulation-based study (Section~\ref{sec:evaluation}) based on trivial small-scale and Internet-like larger-scale topologies.
The results show a great potential of one of the designed algorithms (dynamic limits, Section~\ref{sec:dynamic limits}), where the attack traffic can be effectively localized near the attacker himself.


% Section 2 gives a brief overview of NDN architecture. 
% Section 3 defines Interest flooding attack, 
% Section 4 designes several attack mitigation algorithms.
% Section 5 describes evaluation methodology and gives experimental results. Section 6 contains related work. Section 7 provides analysis based on our results. Section 8 concludes and describes future work.








%%% Local Variables: 
%%% mode: latex
%%% TeX-master: "paper"
%%% End: 


\section{NDN Overview\label{ccn-intro}}
%Named Data Networking (NDN) is a network architecture which aims to replace IP architecture by replacing IP�s host-to-host data delivery model with pull-based information retrieval model. In IP in order to retrieve data, data consumer have to know endpoint location of the desired data, in NDN all they have to know is data name. Names are hierarchical (non-flat) application-defined string names that are used for forwarding, routing and caching. A canonical NDN name looks in a following way: /ndn/ucla/CSdept/faculty/Lixia/webpage.

In this section we briefly introduce NDN with a focus on its stateful forwarding plane (for more details refer to \cite{ndn-conext, ndn-tr, adaptive-forwarding}).
NDN is a receiver-driven, data-centric communication protocol.
All communication in NDN is performed using two distinct types of packets: \textit{Interest} and \textit{Data}. Both types of packets carry a \textit{name}, which uniquely identifies a piece of data that can be carried in one Data packet. Data names in NDN are hierarchically structured. A canonical NDN name looks in the following way: \ndnName{/ndn/ucla/CS/www/index.html}.

To retrieve Data, a consumer puts the name of the desired content into an Interest packet and sends it to the network.
Routers use this name to forward the Interest towards the Data producer, and the Data packet whose name matches the name in the Interest is returned to the consumer.  All Data packets carry a signature that binds the name to the Data.
An Interest is ``satisfied'' when a Data packet is received with matching data name.

Each NDN router maintains three major data structures:
\begin{itemize}
\item \textit{Pending Interest Table (PIT)} holds all ``not yet satisfied'' Interests that have been sent upstream towards data producers. Each entry in PIT contains a list of incoming, and outgoing, physical interfaces; the former enables multicast Data delivery, and the latter is needed when the same Interest is forwarded along multiple paths.
\item \textit{Forwarding Interest Base (FIB)} maps name prefixes to one or multiple physical network interfaces, defining allowed (multipath) directions where to forward Interests. 
% Having one-to-many relationship in this table allows multipath forwarding of Interests.}
\item \textit{Content Store (CS)} temporarily buffers Data packets that pass through this node, allowing fast Data retrieval by different consumers.
\end{itemize}

When a router receives an Interest packet, it first checks whether there is a matching Data in its CS.
If a match is found, the Data is sent back to the incoming interface of the Interest packet.
If not, the Interest name is checked against the entries in the PIT. 
If the name exists in the PIT already, then it can be either a duplicate Interest (i.e., its nonce is remembered in PIT entry) that should be dropped,
or an Interest from another consumer asking for the same Data which requires the incoming interface of this Interest to be added to the existing PIT entry (``collapsing'' the Interests).
If the name does not exist in the PIT, the Interest is added into the PIT and further forwarded to the interface chosen by the strategy module, which uses FIB as input for its decision.

When a Data packet is received, its name is used to look up the PIT.
If a matching PIT entry is found,
the router sends the Data packet to the interface(s) from which the Interest was received, caches the data in the CS, and removes the PIT entry.  Otherwise, the Data packet is unsolicited and discarded. 
Each Interest also has an associated lifetime; the PIT entry is removed when the lifetime expires.
The lifetime is specified by users, but it is ultimately a router's decision for how long it is actually willing to keep the PIT entry.  
For example, and we assume this in the rest of the paper, the maximum time any Interest is kept in PIT is one second.

%Every incoming Interest triggers a lookup in the Content Store and if the Data with matching data name have been found, it is sent back to the same physical interface from which the Interest has arrived. If CS lookup is unsuccessful, Interest must be sent further upstream and put in the PIT. However, all recurring Interests with the same prefix name are stacked together in the same PIT entry and are not sent to the upstream again until this PIT entry expires completely because of timeout. If Data returns from some upstream location it is replicated for each incoming physical interface stored in this PIT entry after that it is removed from PIT. In other words, multipath Data delivery is built-in in NDN. 


%\begin{figure}[htpb]
%  \centering
%  \includegraphics[scale=0.5]{figures/sim-emu-power.pdf}
%  \caption{Strength of Interest flooding attack}
%  \label{fig:simemupower}
%\end{figure}

%\begin{figure}[htpb]
%  \centering
%  \includegraphics[scale=0.5]{figures/sim-emu-performance.pdf}
%  \caption{Data retrieval by legitimate clients}
%  \label{fig:simemuperf}
%\end{figure}


%%% Local Variables: 
%%% mode: latex
%%% TeX-master: "paper"
%%% End: 


\section{Interest Flooding Attacks in NDN}
\label{sec:interest-flooding}



As we explained earlier, Interest packets in NDN are routed through the network based on content name prefixes and consume memory resources at intermediate routers. This makes them a potential tool to launch DDoS attacks in NDN. An attacker or a set of distributed attackers can inject excessive number of Interests in an attempt to overload the network and cause service disruptions for legitimate users (Fig.~\ref{fig:flooding example}). 

\begin{figure}[t]
  \centering
  \includegraphics[scale=0.3]{attack-definition}
  \caption{Example of Interest flooding attack}
  \label{fig:flooding example}
  \vspace{-0.3cm}
\end{figure}


Since an NDN network fetches data by its name, an adversary cannot easily target specific routers or end-hosts.
However, an adversary can target a specific namespace.
For example in Fig.~\ref{fig:flooding example}, if the data producer is the exclusive owner of \ndnName{/foo/bar} namespace, both router B and the data producer would receive all Interests for \ndnName{/foo/bar/\ldots} that cannot be otherwise satisfied from in-network caches.%
\footnote {This example assumes that the adversary floods the network with unique data names carrying \ndnName{\small /foo/bar} prefix to make them effective. It also assumes the producer is single-homed and the data is not replicated elsewhere. With multi-homed producers or replicated data, NDN would likely to cope better with DDoS attacks due its native multipath and adaptive forwarding~\cite{adaptive-forwarding, Yi:2013:A-Case-for-Stateful} support.}
A large volume of such malicious Interests can disrupt service quality in NDN network in two ways: \emph{create network congestion} and \emph{exhaust resources on routers}.

Similar to packets in traditional networks, Interest packets in NDN consume a portion of network capacity. A large number of Interest packets might cause congestion and lead to legitimate packets being dropped in the network. In particular, a coordinated DDoS attack could target one specific namespace and concentrate attack traffic in certain segments of the network, as routing in NDN is based on name prefixes. 


As NDN routers maintain per-packet states for each forwarded Interest (i.e., an entry in its PIT), an excessive amount of malicious Interests can lead to exhaustion of a router's memory, making the router unable to create new PIT entries for incoming Interests and disrupting service for legitimate users.

Nevertheless, creating an effective Interest flooding attack in NDN is non-trivial.
To efficaciously target a specific namespace (e.g., \ndnName{/newyorktimes/}), an adversary needs to make sure that (1)~the expressed Interests are routed towards and as close to the data producer/provider as possible, and (2)~new corresponding PIT entries are created for those interests and are stored at intermediate NDN routers for as long as possible. The former is achieved when Interests share the same name prefix (e.g., \ndnName{/newyorktimes/}) and as long as they are not served from caches of intermediate routers---an Interest is not forwarded upstream if a router can satisfy it from its content store. The latter requires every single malicious Interest to ask for unique content---all Interests requesting the same content are combined into one PIT entry in routers. Thus, an adversary has to request either an unpopular (i.e., not cached in routers) or non-existing unique content with each Interest. Of the two options available to an adversary, the first one is challenging due to the difficulties around indexing content names in a particular namespace, coordinating a large number of bots to send unique Interests, and sustaining the attack while the network is continuously caching the requested content objects. However, the second option---requesting a unique non-existing content with each Interest---is easy to achieve and sustain. For example, an adversary can construct such Interests by concatenating a variable-length random name component to the victim namespace (e.g., \ndnName{/newyorktimes/3rf3...}). In this paper, we exclusively focus on this particular attack strategy as it not only maximizes the damage from each malicious Interest, but also is the one that is easy to launch and widely applicable to all namespaces (small or large). % other less effective strains of Interest flooding attacks can also be mitigated by applying the same or similar countermeasures described in the next section.  

In the rest of this paper, we use the general term \emph{Interest flooding attack} to refer to the above described attack and assume an attacker is limited to controlling a botnet of end-hosts only, i.e., we assume the routers in the network and the computers in the victim domain are not compromised.

%\paragraph{Assumptions} 
%E: These are assumptions for the simulations on hand and particulary to test for the worst case scenario in many aspects. They are not the paper's assumption and in fact the paper first should be more general on describing all possibilities. Then it should explain the assumptions for the simulations and discuss why they make sense and do NOT favor positive results in some way. 

% - assumptions about attacker position
% - assumptions about the producer / producer namespace
% - assumptions about the attack traffic / attack pattern

%In this paper we are making the following assumptions about the Interest flooding attack:
%\begin{itemize}
%\item only client nodes can be compromised and become attack bots;
%\item there are no colluding attackers inside the Data producer's network;
%\item the attack is carried only using unique junk Interests; and
%\item there is only one Data producer for the attacked prefix.
%\end{itemize}

%We also assume that NDN forwarding strategy uses only single-path Interest forwarding, always choosing the best-metric route advertised by the routing protocol.
%This way, we are able to analyze the attack in its best environment, as enabling the multi-path forwarding would only alleviate effects of the Interest flooding attack.

%In section~\ref{sec:discussion} we discuss potential of the Interest flooding attack under several other attack assumptions.


\section{Interest flooding mitigation methods}
\label{sec:design}

% Points what should be here:
% - What can be done (in general) to mitigate flooding attack
% - Which building blocks NDN architecture gives to mitigate Interest flooding attacks: Interest limits, ability to measure Interest satisfaction performance (Interest satisfaction stats)
% - Methods to set these limits: static/dynamic
% - Methods how limits can be applied: best-effort (fifo), "fair" queuing, probabilistic
% - Reference to caching: we don't consider it here, but caching provides additional level of protection, especially for certain types of attacks

% Our definition of attack mitigation is that good clients are still able to access data on the producer.

In this section we present several methods to mitigate Interest flooding attacks in NDN, featuring different degrees of implementation complexity, as well as different degrees of effectiveness.

The presented methods rely on the core NDN principle of the flow balance between Interest and Data packets: Data packet can be forwarded only in response to the incoming Interest and each Interest is satisfied by at most one Data packet.
As a result of this principle, all traffic on NDN is ``receiver''-driven, where the ``receiver'' is either a user or any router on the path between the user and a Data producer.
Therefore, any node can limit the number of forwarded Interests, effectively limiting the amount of the retrieved Data.
At the same time, each forwarded Interest can be used to build up various data plane performance statistics, such as per-incoming interface ratios of satisfied Interests.

% Combining per-incoming interface data plane performance statistics with 

% Disclaimer
% The presented solutions to mitigate Interest flooding DDoS attacks 

% In particular, we explore different mechanisms to app

% Basic limits, probablity based on statistics, and dynamic window adjustment





\subsection{Physical limits}
\label{sec:physical limits}

The requirement to send an Interest in order to receive Data packet, provides an NDN consumer a unique opportunity to request the right amount of Data.
Moreover, the same opportunity to control the amount of data flow is given not only to consumers, but all routers between consumer and producer (or nearby caches).  
In other words, every node, either a consumer or an intermediate router, is able to control how much data it wants to receive by limiting the number of forwarded Interests.

The limitation can implemented in a number of different ways, including leaky bucket scheduling and window-based flow control.
We decided to following TCP-like window-based flow control and applied the sliding window approach to implement Interest limits.

The size of the window defines how many Interests can be send out before Interests get satisfied or expired.
From the one hand, this size should be large enough to ``fill the pipe,'' meaning that a node needs to send enough Interests to receive Data at full capacity of the incoming link.
On the other hand, the window's size should not be too large to avoid excessive buffering and congestion of the Data packet.
Thus, the ideal size for such a window need to be defined proportional to link's bandwidth-delay product~\cite{tcp-survey}.
With the objective to request as many Data packets, as downstream link can pump through, we are getting the following equation for Interest limit:

\begin{equation}
\mathrm{Interest\ Limit} = Delay\ [s] \cdot \frac{\mathrm{Bandwidth\ [Bytes/s]}}{\mathrm{Data\ packet\ size\ [Bytes]}}
\end{equation}

Note that the value of \textit{Delay} is not known a priory and varies between different Interest-Data flows.
However, we do not need to know the exact value of the delay and can set it as an average round trip delay among all flows (with a reasonable filtering of outliers).
This way, the statistical traffic multiplexing with link-level buffering will allow full utilization of the downstream link.
Exactly the same reasoning can be applied to the \textit{Data packet size} parameter, which can also be set to an average observed Data packet size.

Unlike rate-based approaches, window-based limiting does not require precise knowledge about the rate, as well does not need precise scheduling mechanisms.
Like in TCP, the window-based flow is self-clocking, easily adjusting itself to any traffic patterns.

\begin{algorithm}[h]
\caption{Simple physical limits}
\label{alg:simple limits}
\begin{algorithmic}[1]
\For{\textbf{each} interface \em{f}}
    \State $L_{f} \leftarrow$ Interest Limit according to (1)
    \State $O_{f} \leftarrow 0$ \Comment{Outstanding Interests on interface \textbf{f}}
\EndFor

\vspace{0.2cm}
\Function{InInterest}{Interest $i$, InInterface $if$}
\For{\textbf{each} OutInterface $of$ \textbf{in} FwDecision($i,if$)}
    \If{$O_{of} < L_{of}$} \Comment{\textbf{of} is under physical limits}
        \State $O_{of} \leftarrow O_{of} + 1$
        \State Forward($i$, $of$)
    \Else
        \State drop Interest
    \EndIf
\EndFor
\EndFunction

\vspace{0.2cm}

\end{algorithmic}
\end{algorithm}


%%% Local Variables: 
%%% mode: latex
%%% TeX-master: "../paper"
%%% End: 


\subsection{Satisfaction rate statistics generation algorithm}

Statistic generation is performed at different time and name prefix granularities.  
Figure~\ref{fig:stats tree} shows an overall design of the implemented statistics generation module and Pseudocodes~\ref{algo:loadstats}, \ref{algo:interest stats}, \ref{algo:prefix stats}, \ref{algo:stats} show implementation details.

\begin{figure}[htpb]
  \centering
  \includegraphics[scale=0.7]{figures/stats-illustration.pdf}
  \caption{Statistics tree}
  \label{fig:stats tree}
\end{figure}

A separate set of statistical information is kept for each Interest name, as well as aggregated to all Interest name prefixes, including root (``/'') prefix (per-prefix statistics tree on Figure~\ref{fig:stats tree}).

\floatname{algorithm}{Pseudocode}

%%%%%%%%%%%%%%%%%%%%%%%%%%%%%%
%%%%%%%%%%%%%%%%%%%%%%%%%%%%%%
%%%%%%%%%%%%%%%%%%%%%%%%%%%%%%

\begin{algorithm}[h]
\caption{Stats component}
\label{algo:loadstats}
\begin{algorithmic}[1]
\State $\alpha_1 \leftarrow e^{-1.0/5.0}$  \Comment{$\approx$ 5~sec average}
\State $\alpha_2 \leftarrow e^{-1.0/30.0}$ \Comment{$\approx$ 30~sec average}
\State $\alpha_3 \leftarrow e^{-1.0/60.0}$ \Comment{$\approx$ 60~sec average}
\State $avg_1 \leftarrow 0$ \, $avg_2 \leftarrow 0$ \, $avg_3 \leftarrow 0$ \, $counter \leftarrow 0$

\vspace{0.2cm}
\Function{ProcessEvent}{new Event}
\State $counter \leftarrow counter + 1$
\EndFunction

\vspace{0.2cm}
\Function{Advance}{} \Comment{Every second}
\State {} \Comment{\textit{Exponential weighted moving average smoothing}}
\State $avg_1 \leftarrow \alpha_1 \cdot avg_1 + (1 - \alpha_1) \cdot counter$ 
\State $avg_2 \leftarrow \alpha_2 \cdot avg_2 + (1 - \alpha_2) \cdot counter$
\State $avg_3 \leftarrow \alpha_3 \cdot avg_3 + (1 - \alpha_3) \cdot counter$
\State $counter \leftarrow 0$
\EndFunction
\end{algorithmic}
\end{algorithm}

%%%%%%%%%%%%%%%%%%%%%%%%%%%%%%
%%%%%%%%%%%%%%%%%%%%%%%%%%%%%%
%%%%%%%%%%%%%%%%%%%%%%%%%%%%%%

\begin{algorithm}[h]
\caption{Interest Stats component}
\label{algo:interest stats}
\begin{algorithmic}[1]
\State $I \leftarrow$ new Stats   \Comment{\# of Interests}
\State $S \leftarrow$ empty Stats \Comment{\# of satisfied Interests}
\State $U \leftarrow$ empty Stats \Comment{\# of unsatisfied Interests}

% \vspace{0.2cm}
% \Function{ProcessEvent}{new Event}
% \State $I \leftarrow counter + 1$
% \EndFunction

\vspace{0.2cm}
\Function{Combine}{other}
\State $I.counter \leftarrow other.I.counter $ 
\State $S.counter \leftarrow other.S.counter $
\State $U.counter \leftarrow other.U.counter $
\EndFunction

\vspace{0.2cm}
\Function{GetSatisfiedRatio}{}
  \For{each granularity $i$}
     \If{$I.avg_i < 0.1$}
        \State \Return $-1$ \Comment{Unknown state}
     \Else
        \State \Return $\frac{S.avg_i}{I.avg_i}$ \Comment{Interest satisfaction ratio}
     \EndIf
  \EndFor
\EndFunction

\vspace{0.2cm}
\Function{Advance}{} \Comment{Every second}
\State $I.$advance()
\State $S.$advance()
\State $U.$advance()
\EndFunction

\end{algorithmic}
\end{algorithm}

%%%%%%%%%%%%%%%%%%%%%%%%%%%%%%
%%%%%%%%%%%%%%%%%%%%%%%%%%%%%%
%%%%%%%%%%%%%%%%%%%%%%%%%%%%%%

\begin{algorithm}[h]
\caption{Prefix stats component}
\label{algo:prefix stats}
\begin{algorithmic}[1]
\State $T \leftarrow$ empty Interest Stats \Comment{Total per-prefix stats}
\For{\textbf{each} Face f}
    \State $PI[f] \leftarrow$ empty Interest Stats \Comment{per-in Face}
    \State $PO[f] \leftarrow$ empty Interest Stats \Comment{per-out Face}
\EndFor


\vspace{0.2cm}
\Function{NewPitEntry}{}
  \State $T.I.counter \leftarrow T.I.counter + 1$ %\Comment{Total \# of Interets}
\EndFunction

\vspace{0.2cm}
\Function{Incoming}{Face}
  % \State {} \Comment{per-in Face \# of Interests}
  \State $PI[Face].I.counter \leftarrow PI[Face].I.counter + 1$
\EndFunction

\vspace{0.2cm}
\Function{Outgoing}{Face}
  % \State {} \Comment{per-out Face \# of Interests}
  \State $PO[Face].I.counter \leftarrow PO[Face].I.counter + 1$
\EndFunction

\vspace{0.2cm}
\Function{Satisfy}{}
  \State $T.S.counter \leftarrow T.S.counter + 1$ %\Comment{Total \# of Interets}

  \For{\textbf{each} Face $f$}
    % \State {} \Comment{per-in Face \# of Interests}
    \State $PI[f].S.counter \leftarrow PI[f].S.counter + 1$
    % \State {} \Comment{per-out Face \# of Interests}
    \State $PO[f].S.counter \leftarrow PO[f].S.counter + 1$
  \EndFor
\EndFunction

\vspace{0.2cm}
\Function{Timeout}{}
    \State $T.U.counter \leftarrow T.U.counter + 1$ %\Comment{Total \# of Interets}

    \For{\textbf{each} Face $f$}
        % \State {} \Comment{per-in Face \# of Interests}
        \State $PI[f].U.counter \leftarrow PI[f].U.counter + 1$
        % \State {} \Comment{per-out Face \# of Interests}
        \State $PO[f].U.counter \leftarrow PO[f].U.counter + 1$
    \EndFor
\EndFunction

\vspace{0.2cm}
\Function{Combine}{other}
    \State $T.$Combine($other.T$)
    \For{\textbf{each} Face $f$}
        \State $PI[f].$Combine($other.PI[f]$)
        \State $PO[f].$Combine($other.PO[f]$)
    \EndFor
\EndFunction

\vspace{0.2cm}
\Function{Advance}{} \Comment{Every second}
\State $T.$advance()
\For{\textbf{each} Face $f$}
    \State $PI[f].$advance()
    \State $PO[f].$advance()
\EndFor

\EndFunction

\end{algorithmic}
\end{algorithm}

%%%%%%%%%%%%%%%%%%%%%%%%%%%%%%
%%%%%%%%%%%%%%%%%%%%%%%%%%%%%%
%%%%%%%%%%%%%%%%%%%%%%%%%%%%%%

\begin{algorithm}[h]
\caption{Statistics generation algorithm}
\label{algo:stats}
\begin{algorithmic}[1]

\State stats $\leftarrow$ empty tree of Prefix stats \Comment{Stats tree}

\vspace{0.2cm}
\Function{ProcessEvent}{new Interest, inFace}
  \State stats[Interest.Name].NewPitEntry()
  \State stats[Interest.Name].Incoming(inFace);
  \State stats[Interest.Name].Timeout()
\EndFunction

\vspace{0.2cm}
\Function{ProcessEvent}{Interest rejected, inFace}
  \State stats[Interest.Name].NewPitEntry()
  \State stats[Interest.Name].Incoming(inFace)
\EndFunction

\vspace{0.2cm}
\Function{ProcessEvent}{Interest forwarded, outFace}
  \State stats[Interest.Name].Outgoing(outFace)
\EndFunction

\vspace{0.2cm}
\Function{ProcessEvent}{pending Interest timeout}
  \State stats[Interest.Name].Timeout()
\EndFunction

\vspace{0.2cm}
\Function{ProcessEvent}{pending Interest satisfied}
  \State stats[Interest.Name].Satisfy()
\EndFunction

\vspace{0.2cm}
\Function{Periodic}{every second}
\For{every node, walking from children to root}
    \State $node.parent.$Combine($node$)
    \State $node.$Advance()
\EndFor
\EndFunction


\end{algorithmic}
\end{algorithm}

%%% Local Variables: 
%%% mode: latex
%%% TeX-master: "../paper"
%%% End: 


% \subsection{Physical limits with ``fair'' queuing}
% \label{sec:queuing}

% However, this limitation approach does not attempt to utilize data plane performance knowledge (i.e., Interest satisfaction ratio statistics) to discriminate good and malicious Interests.

To partially overcome deficiencies of the Physical limits algorithm we need to ensure that the forwarded Interests represent at least a ``fair'' mix of the Interests received from different neighbors (interfaces).
That is, if the routers A on Fig.~\ref{fig:flooding example} has a very tight token budget, these tokens should be fairly distributed between incoming interfaces \texttt{eth0} and \texttt{eth1}.
Because of the very small volume of Interests, we cannot simply rely on network buffers to do statistical multiplexing of Interests, as they would almost never be buffered.
At the same time, until bag of tokens is not empty, there is no reason to delay Interest forwarding, as we do not known how many and from which interfaces Interests will arrive in the future.
Therefore, in order to achieve the goal of ``fair'' mixing of Interests, we need to implement additional mechanisms to buffer and mix incoming Interests, only if they cannot be immediately forwarded.

For the buffering part, we can reuse Pending Interest Table, with a small extension to support flagging of the Interests that cannot be forwarded immediately (see example on Fig.~\ref{fig:queueing}). 
As for the mixing part, we need an additional fair queuing mechanism, which can be implemented in a form of hierarchical queues (on Fig.~\ref{fig:queueing})\footnote{This essentially is a class based queuing, with classes for each outgoing/incoming interface.} or using virtual time approach~\cite{zhang1990virtual}. 
It should be noted that unlike normal queuing, Interest queues do not actually store a packet, but merely a bi-directional pointer to the existing PIT entry.
This way, PIT entry can be quickly updated when the Interest is actually forwarded, as well as the element can be removed from the queue when the Interest expires.

\begin{figure}[htbp]
  \centering
  \includegraphics[scale=0.65]{queue}
  \caption{Interest queuing: if tokens are unavailable, the router creates PIT entry, but instead of forwarding, enqueued the Interest}
  \label{fig:queueing}
\end{figure}

A more formalized description of the Physical limits algorithms with per-interface fairness is presented in Pseudocode~\ref{alg:queuing}.
The algorithm extends the base Physical limits algorithm by enabling queuing when the bag of tokens is empty (lines 7--10), as well as by triggering an action (lines 16--21), when a token becomes available and enqueued Interest can be finally forwarded.
At the same time, the algorithm limits number of Interests allowed in a queue, constraining memory usage increase by at most a constant factor, compared to the base Physical limits algorithm (i.e., memory attack on routers are still unfeasible).


\floatname{algorithm}{Pseudocode}

%%%%%%%%%%%%%%%%%%%%%%%%%%%%%%
%%%%%%%%%%%%%%%%%%%%%%%%%%%%%%
%%%%%%%%%%%%%%%%%%%%%%%%%%%%%%

\begin{algorithm}[h]
\caption{Physical limits with per-interface fairness}
\label{alg:queuing}
\begin{algorithmic}[1]
\State{} \Comment{Same initialization, InData and Timeout functions as in Physical Limits algorithm}

\vspace{0.2cm}

\Function{OutInterest}{Interest \textbf{i}, InInterface \textbf{if}, OutInterface \textbf{of}}
    \If{$L_{of} - O_{of} > 0$} \Comment{\textbf{of} is under physical limits}
        \State $O_{of} \leftarrow O_{of} + 1$  \Comment{``Borrow'' token}
        \State add \textbf{of} to PIT entry and forward \textbf{i} to \textbf{of}
    \Else
        \State Queue $q \leftarrow of$.GetSubQueue($if$)
        \If{$Size(q) < L_{of}$}
           \State $q$.PushInterest($i$)
           \State add \textbf{of} to PIT entry, and link PIT entry with the queue
        \Else
           \State drop Interest
        \EndIf
    \EndIf
\EndFunction

\vspace{0.2cm}
\State{} \Comment{\textit{Whenever $L_{of} - O_{of}$ becomes larger than zero}}
\Function{TokenBecomesAvailable}{}
    \State Queue $q \leftarrow$ $of$.GetRoundRobinSubQueue 
    \State $i \leftarrow$ $q$.PopInterest
    \State update PIT entry and Forward($i$, $of$)
\EndFunction
\end{algorithmic}
\end{algorithm}


It should be noted that enqueued Interests should not be kept in the queue for a prolonged period of time.
Otherwise, by the time the Interests reaches the Data, the state could have been long expired downstream, effectively making such an Interest useless.
Additional mechanisms of pair-wise agreements between NDN routers and periodic Interest refresh can solve this particular problem, but it is out of the scope of the present paper.

As we show in Section~\ref{sec:evaluation}, fair queueing provides a partial relief from the Interest flooding attack, allowing legitimate users to successfully fetch Data for 15--20\% of the expressed Interests (compared to 0-10\% without fair queueing).
At the same time, the Physical limits with or without fair queueing allows attackers to send a relatively small volume of Interests in order to significantly impact service for the legitimate users.
Therefore, to successfully solve the problem, we need a fundamentally different, more intelligent approach, allowing localization of the attack traffic as close as possible to the attack origin.

%%% Local Variables: 
%%% mode: latex
%%% TeX-master: "../paper"
%%% End: 



Apart from the Interest satisfaction statistics generation, there is a question how this statistics can be used to actually enforce prioritization and penalizing of Interests.
A straightforward way to achieve this enforcement is to consider Interest satisfaction rate as a direct probability to accept (forward) or reject an incoming Interest (see Pseudocode~\ref{alg:probabilistic model}).

\floatname{algorithm}{Pseudocode}

%%%%%%%%%%%%%%%%%%%%%%%%%%%%%%
%%%%%%%%%%%%%%%%%%%%%%%%%%%%%%
%%%%%%%%%%%%%%%%%%%%%%%%%%%%%%

\begin{algorithm}[h]
\caption{Probabilistic model}
\label{alg:probabilistic model}
\begin{algorithmic}[1]
\State{} \Comment{Same initialization, InData and Timeout functions as in Physical Limits algorithm}

\vspace{0.2cm}
\Function{OutInterest}{Interest \textbf{i}, InInterface \textbf{if}, OutInterface \textbf{of}}

    \State{} \Comment{Use uniform probability distribution model $P(X)$}
    \State{} \Comment{$P(X) : \forall x \in [0,1] \Rightarrow P(x) = x$}
    
    \If{$F_{if} > \theta $} \Comment{At least some Interests were forwarded before}
        \State $s \leftarrow (1 - U_{if} / F_{if})$
        \State Drop interest with probability $P(s)$
    \EndIf

    \State{forward the Interest, subjecting to physical limits}
\EndFunction

\end{algorithmic}
\end{algorithm}

Parameter $\theta$ on line 5 of the Pseudocode~\ref{alg:probabilistic model} ensures that the probabilistic model is not applied when there a very small volume of Interests coming from the particular interface.
That is, while we want to deny service to the attackers, we also want give an opportunity for the users who do not abuse the network to regain their share of resources after temporary Data delivery failures.

A main drawback of the probabilistic Interests accept method is the fact that each router on the path make an independent decision on what to do with the Interest.
As a result of such independent decisions, the probability of legitimate Interests not to be dropped decreases quickly as the number of hops between the requester and Data grows, worsening the satisfaction statistics and resulting in further drops.
In example on Fig.~\ref{fig:flooding example}, the router A observes 50\% satisfaction rate for \texttt{eth1} and 0\% rate for \texttt{eth0}. 
At the same time, the router B sees the satisfaction rate for its \texttt{eth0} interface at 30\% level.
Next time a legitimate Interests arrives on the router A, it has 50\% chance to be forwarded further, and if forwarded, it has only $50\% \times 30\% = 15\%$ chance to be actually send towards the Data producer.
To prevent such overreacting, we need to enable some form of a gossiping protocol between neighboring NDN routers, in order to explicitly specify amount of Interests that each router is willing to forward.

%%% Local Variables: 
%%% mode: latex
%%% TeX-master: "../paper"
%%% End: 


% \subsection{Dynamic limits}

\subsection{Dynamic limits assignment algorithm}
\label{sec:dynamic limits}

A notable problem of the queuing method is that the method tries to differentiate between good and malicious Interests, but does not effectively limit the amount of malicious Interests.
That is, if an attacker has a large number of malicious Interests to send, it will receive a slightly lower priority in sending these Interests (some will be delayed more, some number will be dropped), but a lot of these Interests will get through.

One way to solve this problem is announce and dynamically readjust Interest limits for classes (e.g., per prefix and per incoming face), based on Interest satisfaction ratios for those classes. 
Pseudocode~\ref{alg:dynamic limits} summarizes our proposal, which is a variation of a well-known push-back mechanism.


\floatname{algorithm}{Pseudocode}

%%%%%%%%%%%%%%%%%%%%%%%%%%%%%%
%%%%%%%%%%%%%%%%%%%%%%%%%%%%%%
%%%%%%%%%%%%%%%%%%%%%%%%%%%%%%

\begin{algorithm}[h]
\caption{Dynamic limits}
\label{alg:dynamic limits}
\begin{algorithmic}[1]

\Function{AnnounceLimits}{} \Comment{\textit{E.g., every second}}
\For{\textbf{each} prefix $p$ in FIB}
    % \State{} \Comment{Get total limit for the prefix $p$}
    \State $L \leftarrow \displaystyle\sum\limits_{\mathrm{\forall \mathit{of} \in FIB(\mathit{p})}}{}{}of$.GetPerPrefixLimit($p$)
    \State $R \leftarrow 0$
    \For{\textbf{each} available face $f$}
        \State $l_f \leftarrow L \times p$.$f$.GetSatisfactionRatio()
        \State $R \leftarrow R + p$.$f$.GetSatisfactionRatio()
    \EndFor

    \If{$\displaystyle\sum\limits_{\forall f}^{}l_f < l$} \Comment{If under-utilization detected}
        \State $\forall f : l_f \leftarrow \displaystyle\frac{l_f}{R}$ \Comment{Normalize limits}
    \EndIf

        \State AnnounceLimit($f$, $p$, $l$)
\EndFor
\EndFunction

\vspace{0.2cm}

\State{} \Comment{\textit{When announcement from the neighbor is received}}
\Function{InLimits}{InFace $if$, Prefix $p$, Limit $l$}
    \State $if$.GetPerPrefixLimit($p$) $\leftarrow l$ 
\EndFunction

\end{algorithmic}
\end{algorithm}

When an Interest arrives, in addition to be checked against per-outgoing interface limits, it is subjected to per-prefix per-incoming interface limits (lines 3 and 4 in Pseudocode~\ref{alg:dynamic limits}).
In other words, for each prefix in FIB a node cannot send more Interests than a specified limit for this prefix (``dynamic limits''), as well as the node cannot send out more Interests out of the face, than the limit specified for this face (``physical limit'').
While the latter is defined in the same manner as in Section~\ref{sec:physical limits}, ``dynamic limits'' are periodically announced to all neighbors (\textit{AnnounceLimits} function) and adjusted based on received announcements from neighbors (\textit{InLimits} function).%
\footnote{The initial value (not specified in the pseudocode) for the dynamic limits is the corresponding physical limits.}

During the announcement phase, a node tries to determine ``quality'' of neighbors in terms of Interest satisfaction rations and tries to give more resources (accept more Interests from) neighbors with better satisfaction ratios.
At the same time, it is undesirable to over-limit neighbors when there are available resources.
That is, when the sum of all limits calculated by directly applying the satisfaction ratios to the total available limit (lines 15-18 in Pseudocode~\ref{alg:dynamic limits}), limits should be normalized so the sum of all announced limits is at least equal to the total available limit (lines 19-21).

When the satisfaction ratio for a particular prefix and incoming interface becomes close to zero, a node starts announcing (and in real implementation, enforcing) the zero limit.
The node will keep announcing this zero limit for a time period, depending on statistics degradation curve (see more in Section~\ref{sec:stats}).

The dynamic limits method can work as standalone algorithm, as well as it can be (and we believe that it should be, and all results presented in this paper are based on) coupled with the queuing algorithms.
This way, the dynamic limits part pushes back malicious traffic, while queueing part provides limited per-incoming interface fairness for good traffic.


%%% Local Variables: 
%%% mode: latex
%%% TeX-master: "../paper"
%%% End: 


%%% Local Variables: 
%%% mode: latex
%%% TeX-master: "paper"
%%% End: 


\section{Evaluation of Interest flooding mitigation methods}
\label{sec:evaluation}

% List all possible parameters, say clearly which ones we vary, and which ones we do not, along with explanations.

% Metrics that we will consider in our evaluation (Satisfaction rate for good clients, Link utilization near producers, 

% alex: what's a point of latency?  it would matter only for queueing method and we not really pushing this method
% Latency for good clients, good versus bad interests as a function of time

In this section, we present an in-depth evaluation study, aiming to quantify the effectiveness of all our Interest flooding attack mitigation methods.
We used the open-source ndnSIM~\cite{ndnsim} package, which implements NDN protocol stack for NS-3 network simulator (\url{http://www.nsnam.org/}), to run simulations for a variety of network topologies and scenarios. 
We extended ndnSIM with our three mitigation algorithms---token bucket with per interface fairness, satisfaction-based Interest acceptance, or satisfaction-based pushback---and evaluated effectiveness of each algorithm independently.

% For evaluating the effectiveness of each mitigation algorithm, every router in the simulated topology runs the mitigation algorithm under study.

% why we chose small scale topology
% what point we wanted to show we small scale
% what points we wanted to verify with larger-scale internet like topology

% Metrics
% not sure if i already explained that. our essential qualitative metric is .  
% To quantify the effectiveness of the mitigation mechanism this  metric we use satisfaction percentages of user Interests.

The metric we choose to quantify the effectiveness of our algorithms is the {\it percentage of satisfied Interests for legitimate users}. This metric corresponds to the quality of service experienced by legitimate users when the network is under attack. In other words, if the network implements a mitigation method $X$ and a high percentage of user-expressed Interests are satisfied even while the network is under attack, then one can conclude that method $X$ is highly effective at mitigating the attack. We also ensure that all Interests expressed by legitimate users during a period of no-attack are satisfied.
 
%Quality of the attack mitigation methods directly corresponds to the quality of service for the legitimate users during an ongoing attack, which in NDN network can be quantified through percentage of satisfied Interest.
%For example, if the network implements a mitigation method $X$ and under the attack majority of user-expressed Interests are getting satisfied, then the method $X$ can be seen as highly effective.
%At the same time, if only a small percentage of the expressed Interest are getting satisfied during the attack, the method $X$ can be called ineffective. 

% Traffic pattern
In our experiments, we assume that legitimate users express Interests at constant average rates with randomized time gap between two consecutive Interests, where the random number for the gap follows a uniform distribution. We believe that this traffic pattern provides a reasonable approximation of traffic mix from all network users without excessive buffering. To quantify the behavior of our mitigation strategies under a worse case attack scenario, we ensure that all the attackers send junk Interests as fast as they can. 
%(remember, that Interest limits will not allow real flooding of Intersests in all of the designed mitigation algorithms).

% Alex: not sure how to argue about the traffic pattern
%Although this pattern may seem not truly realistic, it approximates a good statistical mix of traffic from all network users without excessive network buffering.

%In addition to that, in each run of the simulation we ensure that all Interests expressed by the legitimate users during a no-attack period are satisfied.
%For simplicity, we also equalized the average rates with which the legitimate users express Interests.
%At the same time, the attackers are sending junk Interests as much as they can get through (remember, that Interest limits will not allow real f%looding of Intersests in all of the designed mitigation algorithms).

We ran our simulations on two different network topologies---a smaller binary tree topology and a much larger ISP-like topology. We use a binary tree topology as it represents one of the worst cases to defend against Interest flooding DDoS attacks. The larger ISP topology reflects how our mitigation methods would perform when deployed on the real Internet.
Again, to study the performance of our mitigation strategies under a range of conditions, we also varied the percentage of attackers in the network---the values ranged from 6\% attackers to over 50\% attackers in the network.

% Fixed parameters
We set the \emph{delay} and \emph{data size} parameters for the Interest limit calculation to a fixed value for every node in the simulated topology. In particular, for the small-scale binary tree topology, we set delay to 80~ms, while for the large-scale ISP topology we set it to 330~ms (the order of the largest RTT). The data size is 1100~bytes for all simulation runs and topologies.

%%%%%%%%%%%%%%%%%%%%%%%%%%%%%%%%%%%%%%%%%%%%%%%%%%
%%%%%%%%%%%%%%%%%%%%%%%%%%%%%%%%%%%%%%%%%%%%%%%%%%

\subsection{Small-scale evaluations}
\label{sec:small-scale}

% Topology description
%To assess baseline quality of the designed Interest flooding attack mitigation methods, we evaluated them first using a %simplistic small-scale binary tree topology 
In Fig.~\ref{fig:small-scale}, we depict the binary-tree topology that we used for our initial experiments.
Legitimate users as well as attackers were placed on leaf nodes (red nodes) as show in the figure. There are 16 end users (both legitimate and attackers) in this topology, each expressing Interests that are routed towards a single data producer placed at the root of the tree.  Each link in this topology is assigned a bandwidth of 10~Mbps and a randomized propagation delay ranging from 1 to 10 ms. 

% Alex: should we mention that?
%The main reason to chose a binary tree topology was that it represents one of the worst cases to defend against flooding DDoS %attacks.
%That is, sharing of the network links exponentially increases as decreasing level of the binary tree.

\begin{figure}[]
  \centering
  \includegraphics[scale=0.2]{topo-tree-evil-5-good-0-producer-gw}
  \caption{Small-scale binary tree topology}
  \label{fig:small-scale}
\end{figure}

% in 10 independent runs of each simulation, where we randomized position of the adversaries along the legitimate users.   In all runs, the total number of legitimate and malicious uses were fixed, meaning that when we increase number of attackers, we decrease the number of legitimate users.   

\subsubsection{Effectiveness of the three mitigation algorithms}

%The first set of experiments aims to evaluate reaction of the network and Interest flooding attack mitigation methods mechanisms under a moderate-level DDoS attack.
%For this purpose, we simulated four different network scenarios, in which all routers implements the same attack mitigation algorithm, either token bucket with per interface fairness, satisfaction-based Interest acceptance, or satisfaction-based pushback (Section~\ref{sec:design}).

Our goal is to compare the effectiveness of each mitigation method and quantify the percentage of Interests satisfied for all legitimate users while the network is under attack. For each mitigation algorithm, we perform 10 independent simulation runs, where we randomly choose 7 client nodes to represent adversaries while the remaining 9 client nodes represent legitimate users. In each run we simulate a 10-minute attack window (total simulation time was 30 minutes, with attack starting at the 10-minute mark). We plot the minimum and maximum range for observed Interest satisfaction percentages for all legitimate users aggregated across the 10 simulation runs as a function of time for each mitigation algorithm in Fig.~\ref{fig:small-scale attack progress}. Token bucket with per-interface fair queuing performs the worst, while satisfaction-based pushback performs the best, with almost a 100\%  satisfied Interests for all legitimate users.

%A short and simplistic summary of the results is that the first two attack mitigation methods do not work at all, and the last two are working quite good.

\begin{figure}[]
  \centering
  \hspace{-0.8cm}\includegraphics[scale=0.8]{paper-topo-tree/tree-good-0-producer-gw}
  \vspace{-.3cm}
  \caption{Interest satisfaction ratio as a function of time for binary-tree topology with 7 attackers and 9 legitimate users}
  \label{fig:small-scale attack progress}
  \vspace{-.4cm}
\end{figure}

%\paragraph{\textbf{Simple token bucket}}

%{\color{red}Alex: should this discussion be removed or we still want to keep it (as it is referenced later and potentially before)}

%Let us take a deeper look on what is happening with the simple token bucket algorithm.
%Essentially, we observe an extremely successful denial of service attack, where attackers almost completely shut down legitimate users from the Data producer (using a relatively small amount of Interests, as token bucket restricts the number of forwarded Intersets!).
%This ``success'' can be explained using a simplistic example, illustrated on Fig.~\ref{fig:three router example}, where each router has only one token for Interest forwarding.
%In this example we assume that both the legitimate user and the adversary send Interests in about the same time.
%
%\begin{figure}[t]
%  \centering
%  \includegraphics[scale=0.3]{physical-limits-sync-problem}
%  \caption{Three-router topology, with one legitimate and one malicious user}
%  \label{fig:three router example}
%\end{figure}
%
%Both router L and router X will forward the Interests, as both of them have a free token available.
%At router A, two cases are possible, either of Interests can arrive first, resulting in a quite different effects.
%When legitimate Interest arrives first, then there are no problem. 
%Router A will capture the token, forward the Interest, which will be quickly satisfied, releasing the token for future uses.
%In the mean time, malicious Interest will be dropped at router A, and router X will release the hold for the token in one second (i.e., after the maximum time Interests are admitted), enabling a new round of competition for router A's resources.
%
%In the case when malicious Interest is sent a little bit earlier, then router A will forward malicious Interests, dropping a legitimate one, and causing ``lock out'' for one second.
%In the instant when the token gets released at router X, an adversary is able to push new Interest towards router A, which may arrive in the exact time A's token gets released (assuming an idealistic environment).
%As a result, the adversary recaptures router A's token and extends the ``lock out'' for another second, denying service to the legitimate user without sending any massive numbers of Interests.

\paragraph{\textbf{Token bucket with per interface fairness}}

We observe a successful DDoS attack, where 40\% attackers succeed in significantly shutting down the remaining 60\% legitimate users---a mere 15\% of their Interests are satisfied by Data from the producer. Contrary to expectations, the 60\% legitimate users do not receive at least 60\% of network resources. As described in the previous section, the key limitation of this algorithm is that it still admits Interests from attackers, a good percentage of which traverse all the way to the data producer. Significantly reducing the number of accepted malicious Interests is fundamental for improving the effectiveness of any mitigation algorithm. 

%The described problem arises from the clocking effect and can be solved in a number different ways.
%Augmenting token bucket algorithms with per-interface fair queueing allows us to eliminate the clocking effect. 
%That is, in the second case when router A releases the token after the first ``lock out'' period, it will immediately process previously enqueued Interests from the legitimate user.
%However, because the Interest time out (``lock out'' time) is most likely be larger than the time to actually fetch the Data, an adversary is still able to ``unfairly'' deny service to good guys.
%Ideally, if there are only 40\% of compromised users, the rest good users should get at least 60\% of network resources, which is not true as can be seen on Fig.~\ref{fig:small-scale attack progress}.
%We expect that reducing the maximum hold time for Interests (e.g., to an order of average RTT) would improve overall performance for legitimate users, with negative effect of requiring extra complexity for Intersets processing.

\paragraph{\textbf{Satisfaction-based Interest acceptance}}

The effectiveness of this algorithm stems from the fact that routers do not admit malicious Interests into the network, thereby ensuring availability of network resources for serving legitimate users. The observed periodic dips in the Interest satisfaction ratios of legitimate users in Fig.~\ref{fig:small-scale attack progress} is a direct result of Interest satisfaction rate statistics decaying with time. The 50-second period approximately corresponds to the selected exponential decaying parameter $\alpha=e^{−1.0/30.0}$, which decays statistics to $1/e$ of the initial value that ranges from 30~seconds to 50~seconds. 
%The primary reason that such minimum peaks exist is the fact that 
When Interests from attackers start to get readmitted, they cause degradation of statistics on routers close to the producer (i.e., routers that observe traffic mix from legitimate and malicious users). Consequently, this degradation reduces the probability of legitimate Interests getting through (see Section~\ref{sec:probabilistic}) until malicious Interests are ``pushed back'' to the edge.

%When routers more intelligently process incoming Interests (i.e., based on the incoming interface statistics), the Interest flooding attack becomes virtually ineffective.
%That is, malicious Interests are simply not getting admitted to the network, not being able to create much service disturbance for the legitimate users.

\paragraph{\textbf{Satisfaction-based pushback}}

This mitigation algorithm is able to effectively shut down attackers and ensure that almost all of the Interests from legitimate users are satisfied. 
%The only potential problem with the satisfaction-based pushback algorithm is that it features 
We observe a  sharp dip in the satisfaction ratio curve at the start of the attack.  It takes a few seconds for all routers to be fully aware of the attack, which happens when malicious Interests start to time out and explicit Interest limit announcements succeed in containing malicious Interests close to the attacker. Till then, the network for a short period of time (under 10~seconds for all simulation runs), fails to provide 100\% service for legitimate users. Once the malicious Interests are effectively shut down, all Interests from legitimate users are satisfied. Unlike the satisfaction-based Interest acceptance scenario, we do not observe any periodic dips in the satisfaction curve, as the pushback algorithm effectively guarantees that once an Interest is admitted, it will almost certainly be routed to the data producer.

\subsubsection{Network reaction with to varying number of attackers}

Our next goal is to study the effectiveness of our mitigation algorithms as a function of increasing adversaries in the network.
To this end, we vary the percentage of attackers in the topology from 6\% to over 50\%. Since the total number of end users in the topology is constant, as the number of attackers increases, the number of legitimate users decreases. All other parameters and experimental set-up are similar to the previous experiment. As before, for each mitigation algorithm, we perform 10 independent simulation runs.
%The second set of conducted experiments was aimed to answer the question of the effect and quality of the Interest flooding attack mitigation algorithm under different attack volume.
%To do this, for each algorithm we varied the number of adversary nodes in the topology, keeping the total number of client nodes constant: 1 attacker and 15 legitimate users, 3 attackers and 13 legitimate users, etc.

%Since at this point the overall attack dynamics of all the attack mitigation algorithms is relatively clear, 
% In other words, each point captured in the box plot graph corresponds to 1-second averaged satisfaction ratio for a user in an individual simulation run.

\begin{figure}[htbp]
  \centering
  \includegraphics[scale=0.8]{paper-topo-tree/tree-good-0-producer-gw-avg-1-min}
  \vspace{-.3cm}
  \caption{Average Interest satisfaction ratios for the first minute of the experiment as a function of increasing attackers in the network}\vspace{-.1cm}
  \label{fig:small-scale-topo boxplot}
\end{figure}

In Fig.\ref{fig:small-scale-topo boxplot} we present  the Interest satisfaction ratio for legitimate users aggregated over the 10 simulation runs for the first minute of the attack. The results are as expected---for all three mitigation algorithms, as the percentage of attackers in the network increases, the lower is the Interest satisfaction ratio for legitimate users.
In the case of token bucket with per-interface fairness algorithm, just 3 attackers can halve the quality of service for the remaining 13 legitimate users. While both intelligent attack mitigation algorithms also show a decline in service quality as the percentage of attackers increases, this decline is much more gradual and marginal. In the case of satisfaction-based pushback algorithm, over 90\% of Interests from legitimate users are satisfied even when over 50\% of the users are malicious.  
 


% \begin{figure}[htbp]
%   \centering
%   \includegraphics[scale=0.9]{tree-topo-var-evils-max-consumers-30mins/tree-good-0-producer-gw-avg-1-min-after-1-min}
%   \caption{Average consumer Interest satisfaction ratios (second minute)}
%   \label{fig:small-scale-topo 2}
% \end{figure}

%%%%%%%%%%%%%%%%%%%%%%%%%%%%%%%%%%%%%%%%%%%
%%%%%%%%%%%%%%%%%%%%%%%%%%%%%%%%%%%%%%%%%%%

\subsection{Large scale simulations}
\label{sec:largescale}

%To check validity of the small-scale experiment results, we performed a larger-scale evaluation based on a modified version of 

Next, our goal is to understand the behavior of our mitigation strategies when deployed on a large-scale real network topology. Our ISP topology is based on a modified version of Rocketfuel's AT\&T topology~\cite{rocketfuel}.
%In order to approximate the general structure of the Internet,
% (scale-free structure, customer-provider, and peer-to-peer relations)
We extract the largest connected component comprising of 562 nodes from this original topology and separate the nodes into three categories: clients, gateways, and backbones. Nodes having degree less than four are classified as clients (344 red nodes as shown in Fig.~\ref{fig:large-scale}), nodes directly connected to clients are classified as gateways (109 green nodes), and the remaining nodes as classified as backbones (109 blue nodes). 
%(To ensure that paths in the topology are ``valley-free,'' we augmented the topology with necessary backbone-to-backbone links.) 
We assign bandwidth and  delay values to links based on their type---both values are random numbers within the respective ranges as shown in Table~\ref{tab:large-scale}. We experiment with placing the data producer at both a gateway node as well as backbone node, which we randomly pick for each simulation run. Similar to the binary tree topology experiments, we fix the number of malicious nodes at approximately 40\% (140 out of 344 client nodes in the topology) and randomly pick these nodes for each simulation run. We conduct 10 simulation runs for each mitigation algorithm, with the attack duration spanning a 5-minute interval.

\begin{figure}[htbp]
  \centering
  \vspace{-.1cm}\includegraphics[scale=0.15,angle=90,height=3.5cm,width=8cm]{7018-r0}
  \caption{Internet-like topology: 344 client routers (red), 109 gateway routers (green), 109 backbone routers (blue)}\vspace{-.2cm}
  \label{fig:large-scale-topo}
\end{figure}

\begin{table}[htbp]
\centering
\caption{Large-scale topology link bandwidth and delay ranges}
\label{tab:large-scale}
\begin{tabular}{|l||c|c||c|c|}
  \hline
  \multirow{2}{*}{\bf Link type} &  \multicolumn{2}{|c||}{\bf Delay} &  \multicolumn{2}{|c|}{\bf Bandwidth} \tabularnewline
  \cline{2-5}
                        &  Min & Max                       &  Min & Max \tabularnewline
  \hline \hline
  Backbone--Backbone    & 5~ms & 10~ms   & 40~Mbps & 100~Mbps \tabularnewline
  \hline
  Gateway--Backbone,    & \multirow{2}{*}{5~ms} & \multirow{2}{*}{10~ms}   
                        & \multirow{2}{*}{10~Mbps} & \multirow{2}{*}{20~Mbps} \tabularnewline
  Gateway--Gateway      & & & & \\
  \hline
  Client--Gateway       & 10~ms & 70~ms   & 1~Mbps  & 3~Mbps \\
  \hline

\end{tabular}
\end{table}

%Priya: Leaving out this in the interest of space...
%Topological location of the data producer plays a key role in its resilience to Interest flooding attacks. For a data producer that %is connected to a client node via a low-bandwidth link, even a small number of junk Interests can impact services for legitimate %users. For a producer located at the backbone with rich connectivity through many high-bandwidth links, an attack might not %be as severe as a majority of legitimate users might not be on the attack path. 


% The results for all attack mitigation algorithms and all runs are aggregated in Fig.~\ref{fig:small-scale attack progress}, where Y-axis represents a minimum and maximum range for observed Interest satisfaction percentages among all nodes and all simulation runs.
% A short and simplistic summary of the results is that the first two attack mitigation methods do not work at all, and the last two are working quite good.

In Fig.~\ref{fig:large-scale}, we summarize our results aggregated over 10 simulations runs for each mitigation algorithm for the scenario, where the data producer is placed at a gateway node. We observe similar results for the data producer placed at the backbone node as well. As in the case of the binary-tree topology experiments, token bucket with per interface fairness is the most ineffective algorithm and satisfaction-based pushback is the most effective one.   Interest satisfaction percentage for legitimate users are approximately 25\% and almost 100\% respectively for these two mitigation methodologies.

%
%The evaluation results,\footnote{Note that for larger-scale experiments we reduced attack window to 15~minutes} summarized in Fig.~\ref{fig:large-scale}, show that performance of the token bucket with per interface fairness and satisfaction-based pushback algorithms are about at the same level as in small-scale evaluations (Fig.~\ref{fig:small-scale}), but with larger variations of minimum and maximum instantaneous satisfaction rates.

\begin{figure}[tbh]
 \centering
 \includegraphics[scale=0.8]{paper-topo-7018-gw/7018-r0-good-0-producer-gw}
 \vspace{-.3cm}\caption{Satisfaction ratio dynamics during the attack for large-scale topology with 40\% attackers)}\vspace{-.4cm}
 % producer on a gateway node
 \label{fig:large-scale}
\end{figure}

Satisfaction-based Interest acceptance algorithm, which was very effective for binary-tree topology, is completely ineffective when deployed in a larger realistic topology. For the duration of the attack, legitimate users experience poor quality of service with only 25\% of their Interests being satisfied and continue to experience degraded service long after the attack has stopped. This poor performance, as detailed in  in Section~\ref{sec:probabilistic}, is due to the fact that each router in the path makes an independent, uncoordinated decision on whether to forward or drop an Interest. In the case of the larger-topology, with much higher average hop count, Interest packets from legitimate users have a very low probability of reaching the data producer, resulting in poor Interest satisfaction statistics, which in turn leads to further penalization of new Interests from them.

To summarize, in all of our simulations, the satisfaction-based pushback algorithm is the most effective technique as it restricts malicious Interests from even entering the network. 
%The only short periods of time when malicious Interests are getting admitted to the network is when routers have either no prior knowledge about per-interface satisfaction ratios (the initial period of the attack) or such knowledge becomes stale (statistics decaying during the attack). As soon as the knowledge is obtained or refreshed, the service for legitimate users returns to norm.


% Alex: Anything else here?

% Alex: I also experimented with placing producer at the backbone, getting slightly better results for all algorithms.  Though I'm not sure there is any value to put those results in the paper

% \subsection{Simulation versus Emulation}
\label{sec:simemu}
Before committing significant efforts into simulation-based implementation of designed defensive techniques it was necessary to confirm that ndnSIM has close performance characteristics to the reference NDN implementation - Project CCNx. This will guarantee that evaluation results derived from simulations will be meaningful in real NDN world.

To achieve this goal a comparison of Project CCNx software and ndnSIM software was performed under small scale Interest flooding attack. DETER Testbed was used as emulation tool for CCNx evaluation. Using it we were able to setup non-virtualized Ubuntu nodes running CCNx 0.6.0 software connected in a binary tree topology with 4 leaves and 1 root node. A number of applications running on top of CCNx have been developed, namely:
\begin{itemize}
\item{Producer application serves 1KB data packets under a known for the attacker name prefix}
\item{Legitimate client application requests 5KB of data per second from the producer}
\item{Attacker application tries to fill the channel of the producer by sending 500 Interest packets per second}
\end{itemize} 

In this emulation scenario producer application occupied a root node, legitimate clients occupied all even leaves and attacker applications were put on all odd leaves. With 100kb links with 40ms delay such setup leads to no congestion during the period when attackers are turned off and congestion when they are turned on (seconds 60-90). Exactly the same scenario was replicated for ndnSIM evaluation, however, we had to adjust the sending rate of attacker application in order to produce the same amount of congestion in the network. Sending rates are compared in Figure~\ref{fig:simemupower}. To achieve the identical slope and height of sending rate of evil Interests by attacker nodes we had to reduce sending rate of simulation-based attacker application by 30\%. The most likely reason for that is the overhead of Java virtual machine and operating system itself during the emulation of CCNx that results in eventual 30\% slower Interest transmission.  

Once we achieved the same characteristics of Interest flooding attack we were able to compare data packet losses by legitimate clients. Figure~\ref{fig:simemuperf} shows the cumulative received data by legitimate consumers in emulation and simulation experiments. NdnSIM performs worse due to its more deterministic nature, while the effects of UDP protocol usage, operating system process scheduling, and other kernel level operations on packet queues provide more randomness and a better intermixing of bad and good traffic which gives a slightly better performance. To summarize, we can use ndnSIM for our evaluations and real world performance is likely to be even better than our evaluation results.

\begin{figure}[htpb]
  \centering
  \includegraphics[scale=0.5]{figures/sim-emu-power.pdf}
  \caption{Strength of Interest flooding attack}
  \label{fig:simemupower}
\end{figure}

\begin{figure}[htpb]
  \centering
  \includegraphics[scale=0.5]{figures/sim-emu-performance.pdf}
  \caption{Data retrieval by legitimate clients}
  \label{fig:simemuperf}
\end{figure}



\subsection{Limitations}
%{\color{red} I gave up the idea of seperate discussion section after trying for some time, as given the space limitation, there is no way to write a reasonably comprehensive and  broad discussion section that justifies a separate section. Instead, I decided to have a scoped down version here that confesses upon the limiations of our evaluations only. Whoever has time (I need to sleep few hours), please go ahead and write 1-2 paragraphs as I outlined below. -Ersin}

%This section should roughly say: Our results show that the two features, namely the symmetric traffic and stateful routing, give NDN routers the ability to observe and characterize traffic in real time. Through our evaluations, we demonstrated that a mitigation algorithm that leverages such ability has great potential in mitigating various DDoS attacks that could otherwise have grave effect on the network. However, as the first step in exploring the problem and the solution space, this paper has obvious limitations: (1) We assume a simple attacker model that would be most effective in vanilla NDN and keep it static during our evaluations. Although it is a valid and important step in showing the effectiveness of a particular mitigation approach, the next natural step is evaluating the above described solutions with more sophisticated and adaptive attackers. (2)   Although most (such as single-path routing, no-caches in the network, single homed victim, etc) are aimed to test the proposed solutions under worst-case scenario, we made many assumptions in simulations/evaluations about the configurations, topology and the traffic that maybe poor-representative of realistic conditions. (3) As NDN being an ongoing research effort itself, our evaluations fully depend on simulations and done over a snapshot of its codebase.       

% The main contribution of the paper and this section in particular is a proof that two inherent properties of NDN architecture, namely the symmetric traffic and stateful forwarding, give NDN routers the ability to characterize traffic and react in real time. 
% We demonstrated that satisfaction-based pushback algorithm, leveraging this ability, has great potential in mitigating various DDoS attacks that could otherwise have a grave effect on the network. 
% However, as the first step in exploring the problem and the solution space, our paper has obvious limitations.
% First, we assumed a simple attacker model that would be most effective in vanilla NDN and keep it static during our evaluations. 
% Although it is a valid and important step in showing the effectiveness of a particular mitigation approach, the next natural step that we are leaving out for future work is evaluating the above described solutions with more sophisticated and adaptive attackers. 
% Second, while aiming to test the designed algorithms under worst-case conditions, the future work needs to address behavior in more realistic scenarios, including investigation of multi-path routing, in network caches, as well as multi-homed victims.

This paper is a first step in understanding the impact of Interest flooding attacks in NDN and exploring the solution space. 
While designing our mitigation algorithms, we exploit two key features of NDN architecture, namely routers maintaining state about the Interests they have forwarded and Data traffic taking the reverse path of the Interest traffic. 
We test the efficacy of our algorithms by simulating them on both a binary-tree topology designed to test the worst-case effectiveness of our algorithms as well as on a realizing ISP-based topology designed to provide insights of our algorithms' behavior when deployed on a real network. 
While our results are promising and show a great potential in mitigating Interest flooding DDoS attacks, there are certain limitations which should be addressed in future research. 

First, in our evaluations we used a simple and static attacker model---attackers send junk Interests as fast as possible. 
In the future work, we plan to explore the impact of models where the attackers are more sophisticated and dynamically adapt their behavior and junk Interest sending patterns based on the network reaction. 
Second, we made an assumption that Interests are not satisfied by an intermediate router's cache and always forwarded all the way to the producer.  
The future work needs to study impact of Interest flooding attack in more realistic scenarios, including investigation effects of multi-path routing and presence of in network caches.
% We also ignored NDN's strategy layer that routes around failure and congestion in the network. 
% In future work, we plan to study the impact of turning on these features. 
Finally, we also plan to perform more extensive simulations on other realistic Internet-like topologies and traffic patterns.



%%% Local Variables: 
%%% mode: latex
%%% TeX-master: "paper"
%%% End: 


\section{Related work \label{related-section}}
In this section we provide a classification of mitigation techniques against DDoS flooding attacks in IP networks, because of the lack of any in NDN or other information centric architectures. 
\begin{itemize}
\item{Capability-based systems} allow routers to negotiate, perform and enforce limitations on bandwidth consumption on router-to-router and router-to-client links. Usually such systems require an extensive trust infrastructure in order to validate secure keys used by routers and clients. This implies that each client must pass through an authentication process prior to using any bandwidth ~\cite{Capabilities}.   
\item{Computation-based systems} provide each client an access to the network resources only after performing a significant computations such as solving puzzles. Spending a huge amounts of computational resources can effectively slow down a flooding attack by the botnet, however it also creates additional computational burden for legitimate clients ~\cite{Portcullis}.
\item{Push back systems} are trying to detect bad and good traffic flows on each router, and once attack is detected by one of the routers, it starts a coordinated push back by downrating incoming flows with a bad traffic in order to provide more capacity for a good traffic. This process is reiterating downstream till it reaches edge routers, which are directly connected to attacking bot machines ~\cite{Pushback}. 
\item{Congestion control systems} do per flow traffic analysis and drop of packets belonging to misbehaving flows. For instance, Random Early Detect (RED)~\cite{RED} identifies flows that do not comply with TCP-friendly end-to-end congestion control, and preferentially drop them. If largely deployed such techniques could perform well, however, they cannot provide effective defense against non-greedy botnets that are creating a huge amount of low-bandwidth flows. 
\end{itemize}

%%% Local Variables: 
%%% mode: latex
%%% TeX-master: "paper"
%%% End: 


\section{Conclusion}
\label{sec:conclusion}


%%% Local Variables: 
%%% mode: latex
%%% TeX-master: "paper"
%%% End: 



\bibliographystyle{IEEEtran}
\bibliography{references}


\end{document}
