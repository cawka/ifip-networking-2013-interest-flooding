\section{Introduction}
\label{sec:intro}

% Introduction / Motivation
% Brief introduction about attacks in NDN
% Contributions of the paper

% Introduction to the problem: ndn, solves many problems, but has potential to introduce new problems.
% in this paer we are studing a number of remedies to Interest floodig attack, as well as trying to doscover ptential of tjese dolutions and their shortcomings

% Structure. Hopefully around three paragraps. 
% - what is happening with ndn and what is interest floodog
% - what are the general approaches we are taking, ranging from naïve to more intelligent
% - basic summary of the results

% move disclaimer from section 3 to the intro.

Various forms of distributed denial of services attack (DDoS) pose a significant and constant threat to the existing Internet infrastructure~\cite{arbor-report}.
The recently proposed Named Data Networking (NDN) architecture~\cite{ndn-conext, ndn-tr} completely changes the communication paradigm in the network by removing the notion of addresses and making data the first-class network entities.
That is, instead of letting end hosts sending packets to a given IP address, NDN lets end hosts request desired data by sending Interest packets carrying application level data names.
Such a shift automatically eliminates several long standing DDoS attacks, including both direct flooding as well as reflector attack through source address spoofing~\cite{mirkovic2004taxonomy}.

However, since NDN allows any host to request any data, malicious users can attack an NDN network by Interest flooding. As we explain in the next section, NDN routers perform stateful forwarding of Interests, uncontrolled floods of Interest packets can potentially exhaust router resources, disrupting services for legitimate users.  
One can also attack an NDN network by hijacking Interest packets and then injecting false data replies into the network.
 
As a first step towards a complete solution to address various denial of service attack problems in NDN, in this paper we investigate the solution space for Interest flooding attack (Section~\ref{sec:interest flooding}).  
We show how an NDN router can exploit its state information to mitigate Interest flooding (Section~\ref{sec:design}).
We base our solution on the following two insights.
First, instead of creating state for all incoming Interests, an NDN router only needs to accept enough Interest packets to fully utilize downstream link capacity.
Second, because Data packets traverse the path taken by the Interests, an NDN router can observe whether an Interest retrieves Data (good Interest) or not (bad Interest).
% Alex: Not sure whether it should be mentioned in intro or not.  If it should, then I don't really like it...
 
We conduct simulation-based evaluations to quantify the effectiveness of our proposed attack mitigation algorithms (Section~\ref{sec:evaluation}).
%% based on trivial small-scale and Internet-like larger-scale topologies.
The results show a great potential of one of the designed algorithms (dynamic limits, Section~\ref{sec:dynamic limits}), where the attack traffic can be effectively localized near the attacker himself.
\todo{LZ: I did not modify this last paragraph; need to see the rest of the paper first.}



%%% Local Variables: 
%%% mode: latex
%%% TeX-master: "paper"
%%% End: 
