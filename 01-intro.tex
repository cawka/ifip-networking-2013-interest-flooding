\section{Introduction}
\label{sec:intro}

% Introduction / Motivation
% Brief introduction about attacks in NDN
% Contributions of the paper

% Introduction to the problem: ndn, solves many problems, but has potential to introduce new problems.
% in this paer we are studing a number of remedies to Interest floodig attack, as well as trying to doscover ptential of tjese dolutions and their shortcomings

% Structure. Hopefully around three paragraps. 
% - what is happening with ndn and what is interest floodog
% - what are the general approaches we are taking, ranging from naïve to more intelligent
% - basic summary of the results

% move disclaimer from section 3 to the intro.

Various forms of distributed denial of services attack (DDoS) pose a significant and constant threat to the existing Internet infrastructure~\cite{arbor-report}.
The recently proposed Named Data Networking (NDN) architecture~\cite{ndn-conext, ndn-tr} completely changes the communication paradigm in the network by removing the notion of network host identities and making the application-level Data names the first-class network citizens.
That is, instead of requiring users to explicitly connect to specific servers and then use application-level protocols to request desired Data, users can directly ask the network for the Data.
% In other words, NDN changes communication from host-centric to data-centric.
Such a shift automatically solves several long standing problems, including source address spoofing and reflector DDoS attacks~\cite{mirkovic2004taxonomy}.

However, as long as there is still a notion of destination, requests (Interests in NDN terms) can still be send towards the destinations,%
\footnote{Note that ```destination'' is much more vague in NDN than in IP. 
Copies of the requested Data, towards which Interests may be forwarded, can be in many places, including long-term and short-term caches on NDN routers and alternative locations of the Data producer.} 
potentially disrupting service to legitimate users.
Taking into account that NDN requires the stateful forwarding of Interets---as the only way for routers to return Data back is to remember from where the Interest came in---an attacker may specifically target the stateful forwarding of NDN, e.g., in a form of Interest flooding attack. 

Our primary goal in this paper is not to completely solve all denial of service problems in NDN, but to demonstrate using the Interest flooding attack example (Section~\ref{sec:interest flooding}) that although new NDN-specific DDoS attacks are possible in NDN network, the network is fully equipped to mitigate effects of these attacks.
In the paper we design several algorithms to mitigate the Interest flooding attack (Section~\ref{sec:design}), exploring the following two unique venues provided by NDN.
First, an NDN router does not need to accept, create state, and forward all the incoming Interests, but only the amount that would fully utilize downstream link capacity.
Second, the fact that Data always follows the Interest paths can be used to distinguish between good (e.g., a forwarded Interest gets satisfied) and ``bad'' Interests. 

% Alex: Not sure whether it should be mentioned in intro or not.  If it should, then I don't really like it...
To quantify the quality of the designed attack mitigation algorithms, we performed a simulation-based study (Section~\ref{sec:evaluation}) based on trivial small-scale and Internet-like larger-scale topologies.
The results show a great potential of one of the designed algorithms (dynamic limits, Section~\ref{sec:dynamic limits}), where the attack traffic can be effectively localized near the attacker himself.


% Section 2 gives a brief overview of NDN architecture. 
% Section 3 defines Interest flooding attack, 
% Section 4 designes several attack mitigation algorithms.
% Section 5 describes evaluation methodology and gives experimental results. Section 6 contains related work. Section 7 provides analysis based on our results. Section 8 concludes and describes future work.








%%% Local Variables: 
%%% mode: latex
%%% TeX-master: "paper"
%%% End: 
