\section{Discussion}
\label{sec:discussion}

% What are the variations of the attack.
% When defining the Interest flooding attack (Section~\ref{sec:interest flooding}) we made a key assumption that adversaries are trying to achieve the maximum harm (network congestion and router overload) to unprotected NDN network by sending a large volume of \emph{junk Interests}.

The attack mitigation methods studied in this paper provide ways to eliminate network congestion and router overload problems caused by junk Interest flooding (Section~\ref{sec:interest flooding}), but unfortunately not without inducing some new problems.
In particular, during an ongoing attack, a number of legitimate Interests successfully forwarded towards the Data producer can be substantially larger in unprotected network, then in the network that implements an Interest mitigation algorithm.  
This is simply because an unprotected network allows clients to pump more packets and does not impose any artificial restrictions on packet forwarding.
Therefore, unless a mitigation algorithm uses some form of intelligence (e.g., statistics for Interest satisfiability), a smaller scale flooding attack can cause greater damage to the service quality.
% (recall service quality curves for physical limits in small-scale and larger-scale evaluations, Fig.~\ref{fig:small-scale} and Fig.~\ref{fig:large-scale}).

Although the studied attack mitigation algorithms base their intelligence on per interface level, it is also possible to use more granular processing.
For example, if routers measure satisfaction ratios and limit Interest forwarding independently for each FIB prefix, then an ongoing attack on one Data namespace will not have any effect ob users requesting Data from another namespace.
 % (assuming that users have to get through te same roiter to get the data)

% (already mentioned in Interest flooding definiton section)
%  Another thing is that rich interconnectivity (de facto on The Internet) and the multi-path packet forwarding (practically impossible) gives NDN addition leverage points to mitigate / reduce effects of ongoing Interest flooding attack of any form.

In the definition of the Interests flooding attack we explicitly stated that an attack is carried using junk Interets, which obviously gives us a clear distinction between ``good'' and ``bad'' Interests.
A curious reader may ask us a valid question: what would happen if an adversary doing something else, e.g., sending sends only ``legitimate'' Interests or intermix them with junk Interests?
Although a complete investigation of this problem deserves a separate study, we anticipate that the problem may not be as severe as it seems.
First, in a general case, the adversaries would not be aware that the network is or is not protected, e.g., is using the satisfaction-based pushback algorithm.
As a result, adversaries would want to do as much harm as possible, which is to send junk Interests.

% Second, when adversary is aware of the attack mitigation mechanisms and sends 
Second, a satisfiable Interests does not cause NDN-specific harm, as they would not linger on routers for extended periods of time, giving opportunities for Interests from the legitimate users to get through and bring Data back.
Essentially, in the worst case scenario (assuming that satisfaction-based pushback is augmented with some form fair queuing), legitimate users and attackers would ``fairly'' share the network.
Note that router resource, as well as upstream link capacity overloading is still impossible, as there is always an upper bound of admitted Interets.

%  and the mitigation algorithm will work as expected: small service quality degradation during first seconds of the attack, followed by practically full recovery from the attack.
% However, a careful reader may ask a totally legitimate question: what would happen if adversaries know that the network is protected and know that it is using the satisfaction-based pushback  algorithm?
% Would there be an optimal strategy to deny service for the legitimate users?
% Would it be better to send ``legitimate'' Interests instead, at least partially, and attempt to always keep the network in the transitional state?

Technically, the ``fair'' sharing of network resources is the only possible way to mitigate the denial of service attacks, if there are no basis to make any distinction between packets, e.g., when attackers are statistically indistinguishable from legitimate users.
At the same time, fairness can be defined in a number different ways, including per interface fairness, per-flow fairness, or even per-computation fairness~\cite{Capabilities, Portcullis}, resulting in different service qualities depending on how the attack is organized.


%%% Local Variables: 
%%% mode: latex
%%% TeX-master: "paper"
%%% End: 
