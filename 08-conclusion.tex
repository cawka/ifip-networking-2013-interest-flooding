\section{Conclusion}
\label{sec:conclusion}

% Main take out from the paper:  could be a problem, but problem is solvable.  
% One of the key components in the Named Data Networking architecture is its stateful data delivery model, which potentially can be abused by new types of denial of service attacks.
% However, one should not forget that the same stateful data delivery provides effective mechanisms to fight back against such attacks.

%In this paper we demonstrated that NDN architecture can be quite resilient against Interest flooding attacks, which aim to overwhelm network capacity as well as memory resources on NDN routers.
%The most effective of the designed algorithms---satisfaction-based pushback---almost completely eliminates the attack (or at least mitigates it to an acceptable ``fair'' level), leveraging two key aspects: Interest/Data path symmetry and full receiver-based traffic control.
%In particular, it measures how well Interests are getting satisfied (path symmetry) and uses these measurements to adjust how many Interests can be accepted from each individual interface (receiver-based control), effectively pushing the attack to the edges.
%
%Other forms of attack may require different mitigation mechanisms.  
%However, the granularity of the NDN state---per each Interest packet---is what giving NDN an opportunity to stand against existing and new forms of attacks.
%
%NDN is a new networking paradigm that shifts the emphasis from endpoints to addressing content directly, thereby enabling secure and efficient content distribution. 

NDN is a newly proposed future Internet architecture, thus it is important to address its resilience in face of DDoS attacks. As a first step in understanding the DDoS threats in NDN, we first examined a specific instance of DDoS attacks, Interest flooding, and the severe service degradation such an attack may cause to legitimate users. We then leverage the key features of the NDN architecture to design, develop, and evaluate three mitigation strategies.
We performed detailed simulations on a range of topologies to quantify the effectiveness of our algorithms.   
The most effective algorithm---satisfaction-based pushback---was successful in almost completely shutting down the attackers so that they cause little or no service impact on legitimate users.

NDN's stateful forwarding plane enables a number of desired functions, such as loop-free, multipath data delivery, built-in multicast, scalable content delivery, effective flow balancing (i.e., congestion avoidance), and robust recovery from network failures, that people have attempted to install in IP networks~\cite{adaptive-forwarding}. Although this useful per-packet state can be abused to launch attacks, the demonstrated success of satisfaction-based pushback algorithm serves as an evidence that one can indeed utilize the per packet state built into each NDN router to enable effective DDoS mitigation as well.  


%<<<<<<< .mine
%NDN is a new networking paradigm that shifts the emphasis from endpoints to addressing content directly, thereby enabling secure and efficient content distribution. However, the resilience of this new architecture to DoS and DDoS attacks has been relatively poorly understood. As a first step in understanding the DDoS threats in NDN, we describe one instance of DDoS attacks called Interest flooding and demonstrate the severe service degradation such an attack can cause to legitimate users. By leveraging and exploiting key features of the NDN architecture design, we also develop and evaluate three mitigation techniques, and perform detailed simulations on a range of topologies to quantify their effectiveness in mitigating interest flooding in NDN . 
%Our first strategy---Token bucket with per interface fairness---while easy to implement and deploy was ineffective at ensuring an acceptable %quality of service for legitimate users.  
%Our most effective algorithm---satisfaction-based pushback---was successful in almost completely eliminating the adverse effects of the attack over the network and resulted in minimal to no service impact for legitimate users. 
%Beyond our success with mitigating the particular attack in NDN, we strongly believe that the lessons from our experience with these mitigation algorithms as well as the metrics that we used to evaluate the effectiveness of our strategies can be applied to develop mitigation strategies for other classes of DDoS attacks in NDN as well.
%=======
%>>>>>>> .r936

%Beyond our success with mitigating Interest flooding attacks in NDN, lessons from our experience with these mitigation algorithms as well as the metrics that we used to evaluate the effectiveness of our strategies can be applied to develop mitigation strategies for other classes of DDoS attacks in NDN. {\color{red} Priya: Need a  strong concluding line on  how it can be used for other kinds of DDoS mitigation strategies. Ersin, can you take a cut at it?}


% We hope that this paper will change attitude to NDN design
% We hope that insights of NDN design potential will 
% Change thinking 
% will spark change in thinking about NDN architecture... 


%%% Local Variables: 
%%% mode: latex
%%% TeX-master: "paper"
%%% End: 
