
\subsection{Intelligent attack mitigation}
\label{sec:intelligent mitigating}

%While the Interest limit is the key building block to suppress mechanisms of the Interest flooding attack, legitimate Interests need to be somehow prioritized and malicious Interests need to be somehow penalized in order to completely suppress the attack.
%That is, instead of processing Interests always based on the first-in-first-serve rule, NDN routers need some basis to treat the incoming Interests differently.
%Thus, the primary task in bringing intelligence to the Interest flooding attack mitigation is to proactively distinguish between legitimate and malicious Interests.

In order to distinguish between legitimate and malicious Interests, we leverage another unique feature of the stateful forwarding in NDN---namely, the guaranteed symmetric flow of Interest and Data packets. Since a Data packet takes the reverse path of the corresponding Interest packet, a router is guarateed to see if an Interest it forwarded resulted in a matching Data packet. The only exception being if Interest/Data packets were lost along the way due to congestion.
Therefore, intermediate routers can classify all Interests that result in matching Data as legitimate, while the ones that timed out can be marked as malicious.\footnote{Recall that in order to maximize effect of the Interest flooding attack, an adversary expresses a large volume of junk Interests (see Section~\ref{sec:interest flooding}).}  
%Implications of other types of attacks are discussed in Section~\ref{sec:discussion}.}

This timeout-based differentiation method is reactive by nature: one cannot know in advance that an Interest would timeout or bring Data back. However, routers can proactively keep an up-to-date statistics of Interests satisfaction ratios (number of forwarded versus number of satisfied Interests), and use this statistic to determine whether an incoming Interest must be forwarded or dropped. For example, maintaining independent Interest satisfaction ratio statistics for each incoming interface is sufficient to estimate whether an Interest received from a neighboring interface will result in matching Data or a timeout. Statistics can also be kept at finer granularities such as per outgoing interface, per name prefix, etc. that can further improve the estimates. A router's goal is to penalize malicious Interests as soon as possible -- negative statistics should build up fast, while positive statistics should not deteriorate too quickly. Choosing the right balance between these contradictory requirements is a challenge and we explore this topic further  in the evaluation section.  {\it Priya: we should address the one that works best such as exponentially weighted moving average in eval section - it doesn't belong here.} 

%The devil is always in the details.
%From the one hand, such statistics needs to start penalizing adversaries as soon as possible (i.e., negative stats should build up fast).
%On the other hand, the positive statistics should not deteriorate too fast (i.e., positive stats should be relatively long-term).
%Our preliminary experiments showed that the standard exponentially weighted moving average, performed once a second with $\alpha$ coefficient $e^{-1/30}$, approximately corresponding to a 30-second averaging window, provides a good balance between the two contradictory requirements.

% \subsubsection{\textbf{Data plane performance tracking}}
% \label{sec:stats}

\subsection{Satisfaction rate statistics generation algorithm}

Statistic generation is performed at different time and name prefix granularities.  
Figure~\ref{fig:stats tree} shows an overall design of the implemented statistics generation module and Pseudocodes~\ref{algo:loadstats}, \ref{algo:interest stats}, \ref{algo:prefix stats}, \ref{algo:stats} show implementation details.

\begin{figure}[htpb]
  \centering
  \includegraphics[scale=0.7]{figures/stats-illustration.pdf}
  \caption{Statistics tree}
  \label{fig:stats tree}
\end{figure}

A separate set of statistical information is kept for each Interest name, as well as aggregated to all Interest name prefixes, including root (``/'') prefix (per-prefix statistics tree on Figure~\ref{fig:stats tree}).

\floatname{algorithm}{Pseudocode}

%%%%%%%%%%%%%%%%%%%%%%%%%%%%%%
%%%%%%%%%%%%%%%%%%%%%%%%%%%%%%
%%%%%%%%%%%%%%%%%%%%%%%%%%%%%%

\begin{algorithm}[h]
\caption{Stats component}
\label{algo:loadstats}
\begin{algorithmic}[1]
\State $\alpha_1 \leftarrow e^{-1.0/5.0}$  \Comment{$\approx$ 5~sec average}
\State $\alpha_2 \leftarrow e^{-1.0/30.0}$ \Comment{$\approx$ 30~sec average}
\State $\alpha_3 \leftarrow e^{-1.0/60.0}$ \Comment{$\approx$ 60~sec average}
\State $avg_1 \leftarrow 0$ \, $avg_2 \leftarrow 0$ \, $avg_3 \leftarrow 0$ \, $counter \leftarrow 0$

\vspace{0.2cm}
\Function{ProcessEvent}{new Event}
\State $counter \leftarrow counter + 1$
\EndFunction

\vspace{0.2cm}
\Function{Advance}{} \Comment{Every second}
\State {} \Comment{\textit{Exponential weighted moving average smoothing}}
\State $avg_1 \leftarrow \alpha_1 \cdot avg_1 + (1 - \alpha_1) \cdot counter$ 
\State $avg_2 \leftarrow \alpha_2 \cdot avg_2 + (1 - \alpha_2) \cdot counter$
\State $avg_3 \leftarrow \alpha_3 \cdot avg_3 + (1 - \alpha_3) \cdot counter$
\State $counter \leftarrow 0$
\EndFunction
\end{algorithmic}
\end{algorithm}

%%%%%%%%%%%%%%%%%%%%%%%%%%%%%%
%%%%%%%%%%%%%%%%%%%%%%%%%%%%%%
%%%%%%%%%%%%%%%%%%%%%%%%%%%%%%

\begin{algorithm}[h]
\caption{Interest Stats component}
\label{algo:interest stats}
\begin{algorithmic}[1]
\State $I \leftarrow$ new Stats   \Comment{\# of Interests}
\State $S \leftarrow$ empty Stats \Comment{\# of satisfied Interests}
\State $U \leftarrow$ empty Stats \Comment{\# of unsatisfied Interests}

% \vspace{0.2cm}
% \Function{ProcessEvent}{new Event}
% \State $I \leftarrow counter + 1$
% \EndFunction

\vspace{0.2cm}
\Function{Combine}{other}
\State $I.counter \leftarrow other.I.counter $ 
\State $S.counter \leftarrow other.S.counter $
\State $U.counter \leftarrow other.U.counter $
\EndFunction

\vspace{0.2cm}
\Function{GetSatisfiedRatio}{}
  \For{each granularity $i$}
     \If{$I.avg_i < 0.1$}
        \State \Return $-1$ \Comment{Unknown state}
     \Else
        \State \Return $\frac{S.avg_i}{I.avg_i}$ \Comment{Interest satisfaction ratio}
     \EndIf
  \EndFor
\EndFunction

\vspace{0.2cm}
\Function{Advance}{} \Comment{Every second}
\State $I.$advance()
\State $S.$advance()
\State $U.$advance()
\EndFunction

\end{algorithmic}
\end{algorithm}

%%%%%%%%%%%%%%%%%%%%%%%%%%%%%%
%%%%%%%%%%%%%%%%%%%%%%%%%%%%%%
%%%%%%%%%%%%%%%%%%%%%%%%%%%%%%

\begin{algorithm}[h]
\caption{Prefix stats component}
\label{algo:prefix stats}
\begin{algorithmic}[1]
\State $T \leftarrow$ empty Interest Stats \Comment{Total per-prefix stats}
\For{\textbf{each} Face f}
    \State $PI[f] \leftarrow$ empty Interest Stats \Comment{per-in Face}
    \State $PO[f] \leftarrow$ empty Interest Stats \Comment{per-out Face}
\EndFor


\vspace{0.2cm}
\Function{NewPitEntry}{}
  \State $T.I.counter \leftarrow T.I.counter + 1$ %\Comment{Total \# of Interets}
\EndFunction

\vspace{0.2cm}
\Function{Incoming}{Face}
  % \State {} \Comment{per-in Face \# of Interests}
  \State $PI[Face].I.counter \leftarrow PI[Face].I.counter + 1$
\EndFunction

\vspace{0.2cm}
\Function{Outgoing}{Face}
  % \State {} \Comment{per-out Face \# of Interests}
  \State $PO[Face].I.counter \leftarrow PO[Face].I.counter + 1$
\EndFunction

\vspace{0.2cm}
\Function{Satisfy}{}
  \State $T.S.counter \leftarrow T.S.counter + 1$ %\Comment{Total \# of Interets}

  \For{\textbf{each} Face $f$}
    % \State {} \Comment{per-in Face \# of Interests}
    \State $PI[f].S.counter \leftarrow PI[f].S.counter + 1$
    % \State {} \Comment{per-out Face \# of Interests}
    \State $PO[f].S.counter \leftarrow PO[f].S.counter + 1$
  \EndFor
\EndFunction

\vspace{0.2cm}
\Function{Timeout}{}
    \State $T.U.counter \leftarrow T.U.counter + 1$ %\Comment{Total \# of Interets}

    \For{\textbf{each} Face $f$}
        % \State {} \Comment{per-in Face \# of Interests}
        \State $PI[f].U.counter \leftarrow PI[f].U.counter + 1$
        % \State {} \Comment{per-out Face \# of Interests}
        \State $PO[f].U.counter \leftarrow PO[f].U.counter + 1$
    \EndFor
\EndFunction

\vspace{0.2cm}
\Function{Combine}{other}
    \State $T.$Combine($other.T$)
    \For{\textbf{each} Face $f$}
        \State $PI[f].$Combine($other.PI[f]$)
        \State $PO[f].$Combine($other.PO[f]$)
    \EndFor
\EndFunction

\vspace{0.2cm}
\Function{Advance}{} \Comment{Every second}
\State $T.$advance()
\For{\textbf{each} Face $f$}
    \State $PI[f].$advance()
    \State $PO[f].$advance()
\EndFor

\EndFunction

\end{algorithmic}
\end{algorithm}

%%%%%%%%%%%%%%%%%%%%%%%%%%%%%%
%%%%%%%%%%%%%%%%%%%%%%%%%%%%%%
%%%%%%%%%%%%%%%%%%%%%%%%%%%%%%

\begin{algorithm}[h]
\caption{Statistics generation algorithm}
\label{algo:stats}
\begin{algorithmic}[1]

\State stats $\leftarrow$ empty tree of Prefix stats \Comment{Stats tree}

\vspace{0.2cm}
\Function{ProcessEvent}{new Interest, inFace}
  \State stats[Interest.Name].NewPitEntry()
  \State stats[Interest.Name].Incoming(inFace);
  \State stats[Interest.Name].Timeout()
\EndFunction

\vspace{0.2cm}
\Function{ProcessEvent}{Interest rejected, inFace}
  \State stats[Interest.Name].NewPitEntry()
  \State stats[Interest.Name].Incoming(inFace)
\EndFunction

\vspace{0.2cm}
\Function{ProcessEvent}{Interest forwarded, outFace}
  \State stats[Interest.Name].Outgoing(outFace)
\EndFunction

\vspace{0.2cm}
\Function{ProcessEvent}{pending Interest timeout}
  \State stats[Interest.Name].Timeout()
\EndFunction

\vspace{0.2cm}
\Function{ProcessEvent}{pending Interest satisfied}
  \State stats[Interest.Name].Satisfy()
\EndFunction

\vspace{0.2cm}
\Function{Periodic}{every second}
\For{every node, walking from children to root}
    \State $node.parent.$Combine($node$)
    \State $node.$Advance()
\EndFor
\EndFunction


\end{algorithmic}
\end{algorithm}

%%% Local Variables: 
%%% mode: latex
%%% TeX-master: "../paper"
%%% End: 


%%%%%%%%%%%%%%%%%%%%%%%
%%%%%%%%%%%%%%%%%%%%%%%
%%%%%%%%%%%%%%%%%%%%%%%

\subsubsection{\textbf{Probabilistic Interest accept}}
\label{sec:probabilistic}


Apart from the Interest satisfaction statistics generation, there is a question how this statistics can be used to actually enforce prioritization and penalizing of Interests.
A straightforward way to achieve this enforcement is to consider Interest satisfaction rate as a direct probability to accept (forward) or reject an incoming Interest (see Pseudocode~\ref{alg:probabilistic model}).

\floatname{algorithm}{Pseudocode}

%%%%%%%%%%%%%%%%%%%%%%%%%%%%%%
%%%%%%%%%%%%%%%%%%%%%%%%%%%%%%
%%%%%%%%%%%%%%%%%%%%%%%%%%%%%%

\begin{algorithm}[h]
\caption{Probabilistic model}
\label{alg:probabilistic model}
\begin{algorithmic}[1]
\State{} \Comment{Same initialization, InData and Timeout functions as in Physical Limits algorithm}

\vspace{0.2cm}
\Function{OutInterest}{Interest \textbf{i}, InInterface \textbf{if}, OutInterface \textbf{of}}

    \State{} \Comment{Use uniform probability distribution model $P(X)$}
    \State{} \Comment{$P(X) : \forall x \in [0,1] \Rightarrow P(x) = x$}
    
    \If{$F_{if} > \theta $} \Comment{At least some Interests were forwarded before}
        \State $s \leftarrow (1 - U_{if} / F_{if})$
        \State Drop interest with probability $P(s)$
    \EndIf

    \State{forward the Interest, subjecting to physical limits}
\EndFunction

\end{algorithmic}
\end{algorithm}

Parameter $\theta$ on line 5 of the Pseudocode~\ref{alg:probabilistic model} ensures that the probabilistic model is not applied when there a very small volume of Interests coming from the particular interface.
That is, while we want to deny service to the attackers, we also want give an opportunity for the users who do not abuse the network to regain their share of resources after temporary Data delivery failures.

A main drawback of the probabilistic Interests accept method is the fact that each router on the path make an independent decision on what to do with the Interest.
As a result of such independent decisions, the probability of legitimate Interests not to be dropped decreases quickly as the number of hops between the requester and Data grows, worsening the satisfaction statistics and resulting in further drops.
In example on Fig.~\ref{fig:flooding example}, the router A observes 50\% satisfaction rate for \texttt{eth1} and 0\% rate for \texttt{eth0}. 
At the same time, the router B sees the satisfaction rate for its \texttt{eth0} interface at 30\% level.
Next time a legitimate Interests arrives on the router A, it has 50\% chance to be forwarded further, and if forwarded, it has only $50\% \times 30\% = 15\%$ chance to be actually send towards the Data producer.
To prevent such overreacting, we need to enable some form of a gossiping protocol between neighboring NDN routers, in order to explicitly specify amount of Interests that each router is willing to forward.

%%% Local Variables: 
%%% mode: latex
%%% TeX-master: "../paper"
%%% End: 


%%%%%%%%%%%%%%%%%%%%%%%
%%%%%%%%%%%%%%%%%%%%%%%
%%%%%%%%%%%%%%%%%%%%%%%

\subsubsection{\textbf{Dynamic limits}}
\label{sec:dynamic limits}

% \subsection{Dynamic limits}

\subsection{Dynamic limits assignment algorithm}
\label{sec:dynamic limits}

A notable problem of the queuing method is that the method tries to differentiate between good and malicious Interests, but does not effectively limit the amount of malicious Interests.
That is, if an attacker has a large number of malicious Interests to send, it will receive a slightly lower priority in sending these Interests (some will be delayed more, some number will be dropped), but a lot of these Interests will get through.

One way to solve this problem is announce and dynamically readjust Interest limits for classes (e.g., per prefix and per incoming face), based on Interest satisfaction ratios for those classes. 
Pseudocode~\ref{alg:dynamic limits} summarizes our proposal, which is a variation of a well-known push-back mechanism.


\floatname{algorithm}{Pseudocode}

%%%%%%%%%%%%%%%%%%%%%%%%%%%%%%
%%%%%%%%%%%%%%%%%%%%%%%%%%%%%%
%%%%%%%%%%%%%%%%%%%%%%%%%%%%%%

\begin{algorithm}[h]
\caption{Dynamic limits}
\label{alg:dynamic limits}
\begin{algorithmic}[1]

\Function{AnnounceLimits}{} \Comment{\textit{E.g., every second}}
\For{\textbf{each} prefix $p$ in FIB}
    % \State{} \Comment{Get total limit for the prefix $p$}
    \State $L \leftarrow \displaystyle\sum\limits_{\mathrm{\forall \mathit{of} \in FIB(\mathit{p})}}{}{}of$.GetPerPrefixLimit($p$)
    \State $R \leftarrow 0$
    \For{\textbf{each} available face $f$}
        \State $l_f \leftarrow L \times p$.$f$.GetSatisfactionRatio()
        \State $R \leftarrow R + p$.$f$.GetSatisfactionRatio()
    \EndFor

    \If{$\displaystyle\sum\limits_{\forall f}^{}l_f < l$} \Comment{If under-utilization detected}
        \State $\forall f : l_f \leftarrow \displaystyle\frac{l_f}{R}$ \Comment{Normalize limits}
    \EndIf

        \State AnnounceLimit($f$, $p$, $l$)
\EndFor
\EndFunction

\vspace{0.2cm}

\State{} \Comment{\textit{When announcement from the neighbor is received}}
\Function{InLimits}{InFace $if$, Prefix $p$, Limit $l$}
    \State $if$.GetPerPrefixLimit($p$) $\leftarrow l$ 
\EndFunction

\end{algorithmic}
\end{algorithm}

When an Interest arrives, in addition to be checked against per-outgoing interface limits, it is subjected to per-prefix per-incoming interface limits (lines 3 and 4 in Pseudocode~\ref{alg:dynamic limits}).
In other words, for each prefix in FIB a node cannot send more Interests than a specified limit for this prefix (``dynamic limits''), as well as the node cannot send out more Interests out of the face, than the limit specified for this face (``physical limit'').
While the latter is defined in the same manner as in Section~\ref{sec:physical limits}, ``dynamic limits'' are periodically announced to all neighbors (\textit{AnnounceLimits} function) and adjusted based on received announcements from neighbors (\textit{InLimits} function).%
\footnote{The initial value (not specified in the pseudocode) for the dynamic limits is the corresponding physical limits.}

During the announcement phase, a node tries to determine ``quality'' of neighbors in terms of Interest satisfaction rations and tries to give more resources (accept more Interests from) neighbors with better satisfaction ratios.
At the same time, it is undesirable to over-limit neighbors when there are available resources.
That is, when the sum of all limits calculated by directly applying the satisfaction ratios to the total available limit (lines 15-18 in Pseudocode~\ref{alg:dynamic limits}), limits should be normalized so the sum of all announced limits is at least equal to the total available limit (lines 19-21).

When the satisfaction ratio for a particular prefix and incoming interface becomes close to zero, a node starts announcing (and in real implementation, enforcing) the zero limit.
The node will keep announcing this zero limit for a time period, depending on statistics degradation curve (see more in Section~\ref{sec:stats}).

The dynamic limits method can work as standalone algorithm, as well as it can be (and we believe that it should be, and all results presented in this paper are based on) coupled with the queuing algorithms.
This way, the dynamic limits part pushes back malicious traffic, while queueing part provides limited per-incoming interface fairness for good traffic.


%%% Local Variables: 
%%% mode: latex
%%% TeX-master: "../paper"
%%% End: 


%%% Local Variables: 
%%% mode: latex
%%% TeX-master: "paper"
%%% End: 


