\begin{abstract}
Distributed Denial of Service (DDoS) attacks are an ongoing problem in today's Internet, where packets from a large number of compromised hosts thwart the paths to the victim site and/or overload the victim machines. 
In a newly proposed future Internet architecture, Named Data Networking (NDN), end users request desired data by sending Interest packets, and the network delivers Data packets upon request only, effectively eliminating many existing DDoS attacks. 
However, an NDN network can be subject to a new type of DDoS attack, namely Interest packet flooding.  
In this paper we investigate effective solutions to mitigate Interest flooding.
We show that NDN's inherent properties of storing per packet state on each router and maintaining flow balance (i.e., one Interest packet retrieves at most one Data packet) provides the  basis for effective DDoS mitigation algorithms.
Our evaluation through simulations shows that the solution can quickly and effectively respond and mitigate Interest flooding.
\end{abstract}

\begin{IEEEkeywords}
Information-centric networks, named-data networking, denial-of-service
\end{IEEEkeywords}
