\section{Evaluation of Interest flooding mitigation methods}
\label{sec:evaluation}

% List all possible parameters, say clearly which ones we vary, and which ones we do not, along with explanations.

% Metrics that we will consider in our evaluation (Satisfaction rate for good clients, Link utilization near producers, 

% alex: what's a point of latency?  it would matter only for queueing method and we not really pushing this method
% Latency for good clients, good versus bad interests as a function of time

In this section, we present an in-depth evaluation study, aiming to quantify the effectiveness of all our Interest flooding attack mitigation methods.
We use the open-source ndnSIM~\cite{ndnsim} package, which implements NDN protocol stack for NS-3 network simulator\footnote{\url{http://www.nsnam.org/}}, to run simulations for a variety of network topologies and scenarios. We implement our three mitigation algorithms---token bucket with per interface fairness, satisfaction-based Interest acceptance, or satisfaction-based pushback---in ndnSIM. For evaluating the effectiveness of each mitigation algorithm, every router in the simulated topology runs the mitigation algorithm under study.

% why we chose small scale topology
% what point we wanted to show we small scale
% what points we wanted to verify with larger-scale internet like topology

% Metrics
% not sure if i already explained that. our essential qualitative metric is .  
% To quantify the effectiveness of the mitigation mechanism this  metric we use satisfaction percentages of user Interests.

The metric we choose to quantify the effectiveness of our defense strategies is the {\it percentage of satisfied Interests for legitimate users}. This metric corresponds to the quality of service experienced by legitimate users when the network is under attack. In other words, if the network implements a mitigation method $X$ and a high percentage of user-expressed Interests are satisfied even while the network is under attack, then one can conlcude that method $X$ is highly effective at mitigating the attack. We also ensure that all Interests expressed by legitimate users during a period of no-attack are satisfied.
 
%Quality of the attack mitigation methods directly corresponds to the quality of service for the legitimate users during an ongoing attack, which in NDN network can be quantified through percentage of satisfied Interest.
%For example, if the network implements a mitigation method $X$ and under the attack majority of user-expressed Interests are getting satisfied, then the method $X$ can be seen as highly effective.
%At the same time, if only a small percentage of the expressed Interest are getting satisfied during the attack, the method $X$ can be called ineffective. 

% Traffic pattern
In our experiments, we assume that legitimate users express Interests at constant average rates with randomized time gap between two consecutive Interests, where the random number for the gap follows a uniform distribution. We believe that this traffic pattern provides a reasonable approximation of traffic mix from all network users without excessive buffering. To quantify the behavior of our mitigation strategies under a worse case attack scenario, we ensure that all the attackers send junk Interests as fast as they can. 
%(remember, that Interest limits will not allow real flooding of Intersests in all of the designed mitigation algorithms).

% Alex: not sure how to argue about the traffic pattern
%Although this pattern may seem not truly realistic, it approximates a good statistical mix of traffic from all network users without excessive network buffering.

%In addition to that, in each run of the simulation we ensure that all Interests expressed by the legitimate users during a no-attack period are satisfied.
%For simplicity, we also equalized the average rates with which the legitimate users express Interests.
%At the same time, the attackers are sending junk Interests as much as they can get through (remember, that Interest limits will not allow real f%looding of Intersests in all of the designed mitigation algorithms).

We ran our simulations on two different network topologies---a smaller binary tree topology and a much larger ISP topology. We use a binary tree topology as it represents one of the worst cases to defend against Interest flooding DDoS attacks. The larger ISP topology reflects how our mitigation methods would perform when deployed on the real Internet.
Again, to study the performance of our mitigation strategies under a range of conditions, we also varied the percentage of attackers in the network - the values ranged from 6\% attackers to over 50\% attackers in the network.

% Fixed parameters
We set the \emph{delay} and \emph{data size} parameters for the Interest limit calculation to a fixed value for every node in the simulated topology. In particular, for the small-scale binary tree topology, we set delay to 80~ms, while for the large-scale ISP topology we set it to 330~ms (the order of the largest RTT). The data size is 1100~bytes for all simulation runs and topologies.


\subsection{Small-scale evaluations}
\label{sec:small-scale}

To assess baseline performance metrics of the designed Interest flooding attack mitigation methods, we evaluated them first using a simplistic small-scale binary tree topology (Fig.~\ref{fig:small-scale}).
All links in this topology were assigned 10~Mbps bandwidths with randomized propagation delays from the range from one to ten milliseconds.
Both legitimate users and attackers were placed only on leaf nodes (red), each expressing Interests towards a single producer placed at the root of the tree. 
% Alex: should we mention that?
The main reason to chose a binary tree topology was that it represents one of the worst cases to defend against flooding DDoS attacks.
That is, sharing of the network links exponentially increases as decreasing level of the binary tree.

\begin{figure}[htbp]
  \centering
  \includegraphics[scale=0.2]{topo-tree-evil-5-good-0-producer-gw}
  \caption{Small-scale binary tree topology}
  \label{fig:small-scale}
\end{figure}


\begin{figure}[htbp]
  \centering
  \includegraphics[scale=1]{tree-topo-var-evils-max-consumers-30mins/tree-good-0-producer-gw-avg-1-min}
  \caption{Average consumer Interest satisfaction ratios (first minute)}
  \label{fig:small-scale-topo 1}
\end{figure}


\begin{figure}[htbp]
  \centering
  \includegraphics[scale=1]{tree-topo-var-evils-max-consumers-30mins/tree-good-0-producer-gw-avg-1-min-after-1-min}
  \caption{Average consumer Interest satisfaction ratios (second minute)}
  \label{fig:small-scale-topo 2}
\end{figure}

\begin{figure*}[t]
  \centering
  \includegraphics[scale=1]{tree-topo-var-evils-max-consumers-30mins/tree-good-0-producer-gw}
  \caption{Satisfaction ratio dynamics during the attack (7 attackers / 9 legitimate)}
  \label{fig:small-scale-topo 3}
\end{figure*}



%%% Local Variables: 
%%% mode: latex
%%% TeX-master: "paper"
%%% End: 

\subsection{Large scale simulations}
\label{sec:largescale}

%To check validity of the small-scale experiment results, we performed a larger-scale evaluation based on a modified version of 

Next, our goal is to understand the behavior of our mitigation strategies when deployed on a large-scale real network topology. Our ISP topology is based on a modified version of Rocketfuel's AT\&T topology~\cite{rocketfuel}.
%In order to approximate the general structure of the Internet,
% (scale-free structure, customer-provider, and peer-to-peer relations)
We extract the largest connected component comprising of 562 nodes from this original topology and separate the nodes into three categories: clients, gateways, and backbones. Nodes having degree than four are classified as clients (344 red nodes as shown in Fig.~\ref{fig:large-scale}), nodes directly connected to clients are classified as gateways (109 green nodes), and the remaining nodes as classified as backbones (109 blue nodes). 
%(To ensure that paths in the topology are ``valley-free,'' we augmented the topology with necessary backbone-to-backbone links.) 
We assign bandwidth and  delay values to links based on their type---both values are random numbers within the respective ranges as shown in Table~\ref{tab:large-scale}. We experiment with placing the data producer at both a gateway node as well as backbone node, which we randomly pick for each simulation run. Similar to the binary tree topology experiments, we fix the number of malicious nodes at approximately 40\% (140 out of 344 client nodes in the topology) and randomly pick these nodes for each simulation run. We conduct 10 simulation runs for each mitigation algorithm, with the attack duration spanning a 5 minute interval.

\begin{figure}[htbp]
  \centering
  \vspace{-.1cm}\includegraphics[scale=0.15,angle=90]{7018-r0}
  \caption{Internet-like topology: 344 client routers (red), 109 gateway routers (green), 109 backbone routers (blue)}\vspace{-.2cm}
  \label{fig:large-scale-topo}
\end{figure}

\begin{table}[htbp]
\centering
\caption{Large-scale topology link bandwidth and delay ranges}
\label{tab:large-scale}
\begin{tabular}{|l||c|c||c|c|}
  \hline
  \multirow{2}{*}{\bf Link type} &  \multicolumn{2}{|c||}{\bf Delay} &  \multicolumn{2}{|c|}{\bf Bandwidth} \tabularnewline
  \cline{2-5}
                        &  Min & Max                       &  Min & Max \tabularnewline
  \hline \hline
  Backbone--Backbone    & 5~ms & 10~ms   & 40~Mbps & 100~Mbps \tabularnewline
  \hline
  Gateway--Backbone,    & \multirow{2}{*}{5~ms} & \multirow{2}{*}{10~ms}   
                        & \multirow{2}{*}{10~Mbps} & \multirow{2}{*}{20~Mbps} \tabularnewline
  Gateway--Gateway      & & & & \\
  \hline
  Client--Gateway       & 10~ms & 70~ms   & 1~Mbps  & 3~Mbps \\
  \hline

\end{tabular}
\end{table}

%Priya: Leaving out this in the interest of space...
%Topological location of the data producer plays a key role in its resilience to Interest flooding attacks. For a data producer that %is connected to a client node via a low-bandwidth link, even a small number of junk Interests can impact services for legitimate %users. For a producer located at the backbone with rich connectivity through many high-bandwidth links, an attack might not %be as severe as a majority of legitimate users might not be on the attack path. 


% The results for all attack mitigation algorithms and all runs are aggregated in Fig.~\ref{fig:small-scale attack progress}, where Y-axis represents a minimum and maximum range for observed Interest satisfaction percentages among all nodes and all simulation runs.
% A short and simplistic summary of the results is that the first two attack mitigation methods do not work at all, and the last two are working quite good.

In Fig.~\ref{fig:large-scale}, we summarize our results aggregated over 10 simulations runs for each mitigation algorithm for the scenario where the data producer is placed at a gateway node. We observe similar results for the data producer placed at the backbone node as well. As in the case of the binary-tree topology experiments, token bucket with per interface fairness is the most ineffective algorithm and satisfaction-based pushback is the most effective one.   Interest satisfaction percentage for legitimate users are approximately 25\% and almost 100\% respectively for these two mitigation methodologies.

%
%The evaluation results,\footnote{Note that for larger-scale experiments we reduced attack window to 15~minutes} summarized in Fig.~\ref{fig:large-scale}, show that performance of the token bucket with per interface fairness and satisfaction-based pushback algorithms are about at the same level as in small-scale evaluations (Fig.~\ref{fig:small-scale}), but with larger variations of minimum and maximum instantaneous satisfaction rates.

\begin{figure}[tbh]
 \centering
 \includegraphics[scale=0.8]{paper-topo-7018-gw/7018-r0-good-0-producer-gw}
 \vspace{-.7cm}\caption{Satisfaction ratio dynamics during the attack for large-scale topology with 40\% attackers)}\vspace{-.4cm}
 % producer on a gateway node
 \label{fig:large-scale}
\end{figure}

Satisfaction-based Interest acceptance algorithm, which was very effective for binary-tree topology, is completely ineffective when deployed in a larger realistic topology. For the duration of the attack, legitimate users experience poor quality of service with only 25\% of their Interests being satisfied and continue to experience degraded service long after the attack has stopped. This poor performance, as detailed in  in Section~\ref{sec:probabilistic}, is due to the fact that each router in the path makes an independent, uncoordinated decision on whether to forward or drop an Interest. In the case of the larger-topology, with much higher average hop count, Interest packets from legitimate users have a very low probability of reaching the data producer, resulting in poor Interest satisfaction statistics, which in turn leads to further penalization of new Interests from them.

To summarize, in all of our simulations, the satisfaction-based pushback algorithm is the most effective technique as it restricts malicious Interests from even entering the network. 
%The only short periods of time when malicious Interests are getting admitted to the network is when routers have either no prior knowledge about per-interface satisfaction ratios (the initial period of the attack) or such knowledge becomes stale (statistics decaying during the attack). As soon as the knowledge is obtained or refreshed, the service for legitimate users returns to norm.


% Alex: Anything else here?

% Alex: I also experimented with placing producer at the backbone, getting slightly better results for all algorithms.  Though I'm not sure there is any value to put those results in the paper


%%% Local Variables: 
%%% mode: latex
%%% TeX-master: "paper"
%%% End: 


% \subsection{Simulation versus Emulation}
\label{sec:simemu}
Before committing significant efforts into simulation-based implementation of designed defensive techniques it was necessary to confirm that ndnSIM has close performance characteristics to the reference NDN implementation - Project CCNx. This will guarantee that evaluation results derived from simulations will be meaningful in real NDN world.

To achieve this goal a comparison of Project CCNx software and ndnSIM software was performed under small scale Interest flooding attack. DETER Testbed was used as emulation tool for CCNx evaluation. Using it we were able to setup non-virtualized Ubuntu nodes running CCNx 0.6.0 software connected in a binary tree topology with 4 leaves and 1 root node. A number of applications running on top of CCNx have been developed, namely:
\begin{itemize}
\item{Producer application serves 1KB data packets under a known for the attacker name prefix}
\item{Legitimate client application requests 5KB of data per second from the producer}
\item{Attacker application tries to fill the channel of the producer by sending 500 Interest packets per second}
\end{itemize} 

In this emulation scenario producer application occupied a root node, legitimate clients occupied all even leaves and attacker applications were put on all odd leaves. With 100kb links with 40ms delay such setup leads to no congestion during the period when attackers are turned off and congestion when they are turned on (seconds 60-90). Exactly the same scenario was replicated for ndnSIM evaluation, however, we had to adjust the sending rate of attacker application in order to produce the same amount of congestion in the network. Sending rates are compared in Figure~\ref{fig:simemupower}. To achieve the identical slope and height of sending rate of evil Interests by attacker nodes we had to reduce sending rate of simulation-based attacker application by 30\%. The most likely reason for that is the overhead of Java virtual machine and operating system itself during the emulation of CCNx that results in eventual 30\% slower Interest transmission.  

Once we achieved the same characteristics of Interest flooding attack we were able to compare data packet losses by legitimate clients. Figure~\ref{fig:simemuperf} shows the cumulative received data by legitimate consumers in emulation and simulation experiments. NdnSIM performs worse due to its more deterministic nature, while the effects of UDP protocol usage, operating system process scheduling, and other kernel level operations on packet queues provide more randomness and a better intermixing of bad and good traffic which gives a slightly better performance. To summarize, we can use ndnSIM for our evaluations and real world performance is likely to be even better than our evaluation results.

\begin{figure}[htpb]
  \centering
  \includegraphics[scale=0.5]{figures/sim-emu-power.pdf}
  \caption{Strength of Interest flooding attack}
  \label{fig:simemupower}
\end{figure}

\begin{figure}[htpb]
  \centering
  \includegraphics[scale=0.5]{figures/sim-emu-performance.pdf}
  \caption{Data retrieval by legitimate clients}
  \label{fig:simemuperf}
\end{figure}



%%% Local Variables: 
%%% mode: latex
%%% TeX-master: "paper"
%%% End: 
