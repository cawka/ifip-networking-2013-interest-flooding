\section{NDN Overview\label{sec:ccn-intro}}

In this section we briefly introduce NDN with a focus on its stateful forwarding plane (refer to \cite{ndn-conext, ndn-tr, adaptive-forwarding, Yi:2013:A-Case-for-Stateful} for more details).
NDN is a receiver-driven, data-centric communication protocol.
All communications in NDN are performed using two distinct types of packets: \textit{Interest} and \textit{Data}. Both types of packets carry a \textit{name}, which uniquely identifies a piece of content that can be carried in one Data packet. Data names in NDN are hierarchically structured and an example name for the first segment of a youtube video would look like: \ndnName{/youtube/videos/0F8YdlkKO9A/0}.

To retrieve data, a consumer requests it by sending an Interest packet with the name of the desired content in it.
Routers use this name to route the Interest towards data sources, and a Data packet whose name matches the name in the Interest is returned to the consumer by following the reverse path of the Interest. Similar to IP, Interest forwarding is based on longest name prefix match, but, unlike IP, an Interest packet and its matching Data packet always take symmetric paths.
%% An Interest is ``satisfied'' when it is matched with a Data packet.

Each NDN router maintains three major data structures:
\begin{itemize}
\item \textit{Pending Interest Table (PIT)} holds all ``not yet satisfied'' Interests that have been sent upstream towards potential data sources. Each PIT entry contains one or multiple incoming and outgoing physical interfaces; multiple incoming interfaces indicate the same data is requested from multiple downstream users; multiple outgoing interfaces indicate the same Interest is forwarded along multiple paths.
\item \textit{Forwarding Interest Base (FIB)} maps name prefixes to one or multiple physical network interfaces, specifying directions where Interests can be forwarded. 
% Having one-to-many relationship in this table allows multipath forwarding of Interests.}
\item \textit{Content Store (CS)} temporarily buffers Data packets that pass through this router, allowing efficient data retrieval by different consumers.
\end{itemize}

When a router receives an Interest packet, it first checks whether there is a matching data in its CS.
If a match is found, the Data packet is sent back to the incoming interface of the Interest packet.
If not, the Interest name is checked against the entries in the PIT. 
If the name already exists in the PIT, then it can be a duplicate Interest (identified by a random number each Interest carries) that should be dropped,
or an Interest from another consumer asking for the same Data, which requires the incoming interface of this Interest to be added to the existing PIT entry.
If the name does not exist in the PIT, the Interest is added into the PIT and forwarded along the interface chosen by the strategy module, which uses FIB as input for its decisions.

When a Data packet is received, its name is used to look up the PIT.
If a matching PIT entry is found,
the router sends the Data packet to the interface(s) from which the Interest was received, caches the data in the CS, and removes the PIT entry.  Otherwise, the Data packet is deemed unsolicited and is discarded. 
Each Interest also has an associated lifetime; the PIT entry is removed when the lifetime expires.
Although the maximum lifetime is specified by users, it is ultimately a router's decision on how long it is willing to keep a PIT entry.  
For simplicity, we assume routers keep Interests in PIT for one second in our simulations.

