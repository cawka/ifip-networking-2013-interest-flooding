
Apart from the Interest satisfaction statistics generation, there is a question how this statistics can be used to actually enforce prioritization and penalizing of Interests.
A straightforward way to achieve this enforcement is to consider Interest satisfaction rate as a direct probability to accept (forward) or reject an incoming Interest (see Pseudocode~\ref{alg:probabilistic model}).

\floatname{algorithm}{Pseudocode}

%%%%%%%%%%%%%%%%%%%%%%%%%%%%%%
%%%%%%%%%%%%%%%%%%%%%%%%%%%%%%
%%%%%%%%%%%%%%%%%%%%%%%%%%%%%%

\begin{algorithm}[h]
\caption{Probabilistic model}
\label{alg:probabilistic model}
\begin{algorithmic}[1]
\State{} \Comment{Same initialization, InData and Timeout functions as in Physical Limits algorithm}

\vspace{0.2cm}
\Function{OutInterest}{Interest \textbf{i}, InInterface \textbf{if}, OutInterface \textbf{of}}

    \State{} \Comment{Use uniform probability distribution model $P(X)$}
    \State{} \Comment{$P(X) : \forall x \in [0,1] \Rightarrow P(x) = x$}
    
    \If{$F_{if} > \theta $} \Comment{At least some Interests were forwarded before}
        \State $s \leftarrow (1 - U_{if} / F_{if})$
        \State Drop interest with probability $P(s)$
    \EndIf

    \State{forward the Interest, subjecting to physical limits}
\EndFunction

\end{algorithmic}
\end{algorithm}

Parameter $\theta$ on line 5 of the Pseudocode~\ref{alg:probabilistic model} ensures that the probabilistic model is not applied when there a very small volume of Interests coming from the particular interface.
That is, while we want to deny service to the attackers, we also want give an opportunity for the users who do not abuse the network to regain their share of resources after temporary Data delivery failures.

A main drawback of the probabilistic Interests accept method is the fact that each router on the path make an independent decision on what to do with the Interest.
As a result of such independent decisions, the probability of legitimate Interests not to be dropped decreases quickly as the number of hops between the requester and Data grows, worsening the satisfaction statistics and resulting in further drops.
In example on Fig.~\ref{fig:flooding example}, the router A observes 50\% satisfaction rate for \texttt{eth1} and 0\% rate for \texttt{eth0}. 
At the same time, the router B sees the satisfaction rate for its \texttt{eth0} interface at 30\% level.
Next time a legitimate Interests arrives on the router A, it has 50\% chance to be forwarded further, and if forwarded, it has only $50\% \times 30\% = 15\%$ chance to be actually send towards the Data producer.
To prevent such overreacting, we need to enable some form of a gossiping protocol between neighboring NDN routers, in order to explicitly specify amount of Interests that each router is willing to forward.

%%% Local Variables: 
%%% mode: latex
%%% TeX-master: "../paper"
%%% End: 
