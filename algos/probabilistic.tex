
In addition to maintaining statistics on Interest satisfaction ratios, the next challenge is in using these statistics to penalize malicious Interests. A straightforward method to achieve this enforcement is to consider Interest satisfaction rate as a direct probability for accepting (forwarding) or rejecting an incoming Interest (see Pseudocode~\ref{alg:probabilistic model}).
%Apart from the Interest satisfaction statistics generation, there is a question how this statistics can be used to actually enforce prioritization and penalizing of Interests.


\floatname{algorithm}{Pseudocode}

%%%%%%%%%%%%%%%%%%%%%%%%%%%%%%
%%%%%%%%%%%%%%%%%%%%%%%%%%%%%%
%%%%%%%%%%%%%%%%%%%%%%%%%%%%%%

\begin{algorithm}[h]
\footnotesize
\caption{\small Satisfaction-based Interest acceptance}
\label{alg:probabilistic model}
\begin{algorithmic}[1]
\State{} \Comment{Same init, InData and Timeout functions as in Pseudocode~\ref{alg:queuing}}

\vspace{0.1cm}
\Function{OutInterest}{Interest \textbf{i}, InInterface \textbf{in}, OutInterface \textbf{out}}

    \State{} \Comment{Use uniform probability distribution model $P(X)$}
    \State{} \Comment{$P(X) : \forall x \in [0,1] \Rightarrow P(x) = x$}
    
    \If{$F_{in} > \theta $} \Comment{At least some Interests were forwarded before}
        \State $s \leftarrow (1 - U_{in} / F_{in})$
        \State Drop interest with probability $P(s)$
    \EndIf

    \State{forward the Interest, subjecting to token bucket limits}
\EndFunction

\end{algorithmic}
\end{algorithm}

Parameter $\theta$ on line 5 of the Pseudocode~\ref{alg:probabilistic model} ensures that the probabilistic model is not enforced when the volume of Interests arriving at a particular interface is small.
This step is critical---while we want to drop Interests from attackers, we also want to provide an opportunity for legitimate users to regain their share of resources after temporary Data delivery failures due to  congestion.

A drawback of the satisfaction-based Interest acceptance method is that each router on the path makes an independent decision on whether to forward or drop the Interest. 
As a result of these independent decisions,  the probability of legitimate Interests being forwarded decreases rapidly as the number of hops between the content requester and producer grows; worsening the Interest satisfaction statistics and resulting in further drops.
In example on Fig.~\ref{fig:flooding example}, the router A observes 50\% satisfaction rate for \texttt{eth1} and 0\% rate for \texttt{eth0}. 
At the same time, router B observes a 30\% satisfaction rate for its \texttt{eth0} interface.
Next time a legitimate Interest arrives at router A, it has a 50\% chance of being forwarded further, and if forwarded, it has only a $50\% \times 30\% = 15\%$ probability of being forwarded further towards the Data producer. With each increasing hop in the network, the probability of being forwarded to the next hop decreases significantly. 
One way to prevent this overreaction and unfair penalization is to ensure that the decision taken at each router on whether to forward or drop the Interest is not independent of the decision taken at preceding routers. An explicit notification such as a gossip protocol between neighboring NDN routers that specifies the volume of Interests each router is willing to forward will likely address this issue.
{\color{red}Alex: we should state that it is out of scope of the paper to evaluate this issue}

%%% Local Variables: 
%%% mode: latex
%%% TeX-master: "../paper"
%%% End: 
