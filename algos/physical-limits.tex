\subsection{Physical limits}
\label{sec:physical limits}

The requirement to send an Interest in order to receive Data packet, provides an NDN consumer a unique opportunity to request the right amount of Data.
Moreover, the same opportunity to control the amount of data flow is given not only to consumers, but all routers between consumer and producer (or nearby caches).  
In other words, every node, either a consumer or an intermediate router, is able to control how much data it wants to receive by limiting the number of forwarded Interests.

The limitation can implemented in a number of different ways, including leaky bucket scheduling and window-based flow control.
We decided to following TCP-like window-based flow control and applied the sliding window approach to implement Interest limits.

The size of the window defines how many Interests can be send out before Interests get satisfied or expired.
From the one hand, this size should be large enough to ``fill the pipe,'' meaning that a node needs to send enough Interests to receive Data at full capacity of the incoming link.
On the other hand, the window's size should not be too large to avoid excessive buffering and congestion of the Data packet.
Thus, the ideal size for such a window need to be defined proportional to link's bandwidth-delay product~\cite{tcp-survey}.
With the objective to request as many Data packets, as downstream link can pump through, we are getting the following equation for Interest limit:

\[
\mathrm{Interest\ Limit} = Delay\ [s] \cdot \frac{\mathrm{Bandwidth\ [Bytes/s]}}{\mathrm{Data\ packet\ size\ [Bytes]}}
\]

Note that the value of \textit{Delay} is not known a priory and varies between different Interest-Data flows.
However, we do not need to know the exact value of the delay and can set it as an average round trip delay among all flows (with a reasonable filtering of outliers).
This way, the statistical traffic multiplexing with link-level buffering will allow full utilization of the downstream link.
Exactly the same reasoning can be applied to the \textit{Data packet size} parameter, which can also be set to an average observed Data packet size.

Unlike rate-based approaches, window-based limiting does not require precise knowledge about the rate, as well does not need precise scheduling mechanisms.
Like in TCP, the window-based flow is self-clocking, easily adjusting itself to any traffic patterns.

%%% Local Variables: 
%%% mode: latex
%%% TeX-master: "../paper"
%%% End: 
