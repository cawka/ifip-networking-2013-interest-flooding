\section{Related work}
\label{sec:related-work}

As a new Internet architecture proposal, the Named Data Networking has attracted attention from the security community. 
Lauinger~\cite{Lauinger:2010:Security--scalability} showed several possible attacks against some specific design choices implemented in CCNx software (\url{http://www.ccnx.org/}); in particular~\cite{Lauinger:2010:Security--scalability} explored the issue of user privacy, assuming users are concentrated on the same edge router only and one can obtain complete knowledge of cached content.  
W\"ahlisch et al.~\cite{Wahlisch:2012:Backscatter-from} explore security and stability threats in the general area of information-centric networks (ICN), using CCNx software as an example.  
While this work aims to generalize findings for all ICNs, it largely does not go beyond the current design choices of CCNx.
% (which is an example, but not the final word in NDN) 
% Gasti et al.~\cite{gasti2012ddos} identified several NDN architecture-specific attacks (Interest flooding and content/cache poisoning) and proposed potential solutions, without devising particular algorithms.
In this paper we follow the suggestions by Gasti et al.~\cite{gasti2012ddos} and utilize the properties provided by the NDN architecture to mitigate DDoS attacks, and tackle the Interest flooding problem as a first step in this direction.
More specifically our attack mitigation design relies on NDN's stateful forwarding plane 
which provides statistics on unsatisfied Interests.  
Similar approaches are also explored by Yi et al.~\cite{adaptive-forwarding,adaptive-tr} and Rozhnova et al.~\cite{rozhnova2012effective} to facilitate NDN network performance.

The general area of mitigating denial of service attacks is among the hottest topics in recent years.
% Among all the body of work, 
The most relevant works to this paper are the solutions to brute force IP packet flooding, such as DDoS detection techniques~\cite{Chen:2007:Collaborative-detection, Jun:2011:DDoS-flooding}, methods of pushbacking malicious traffic to edges using various filtering techniques~\cite{Pushback, Tupakula:2003:A-practical-method, Argyraki:2005:Active-internet, Oikonomou:2006:A-framework-for-a-collaborative, Liu:2008:To-filter-or-to-authorize:, Chou:2009:Proactive-surge, Liu:2010:NetFence:-preventing}, 
overlay-based filtering~\cite{Stone:2000:CenterTrack:-An-IP-overlay, Keromytis:2004:SOS:-An-architecture, Kline:2011:Shield:-DoS-filtering}, 
various ticketing systems with conditional admission of traffic to the network~\cite{Yaar:2004:SIFF:-A-stateless, Yang:2005:A-DoS-limiting-network, Wendlandt:2006:Fastpass:-Providing, Natu:2007:Fine-grained-capabilities, Portcullis, Capabilities},
and systems that attempt to use approximate IP traffic symmetry to estimate and filter out malicious traffic~\cite{Wang:2002:Detecting-SYN-flooding, Kreibich:2005:Using-packet, Mahimkar:2007:dFence:-Transparent}. 
Identifying malicious traffic requires establishing certain state at routers, and the above mentioned solutions differ in the specifics on what state to use and how to establish it.  By design NDN's forwarding plane keeps per packet state at every router, the finest granularity state to support any and all mitigation solutions. Therefore, one can simply take full advantage of this state to develop desired solutions.

% Our design for the Interest flooding attack mitigation borrows 

% This paper talks about attacks on content centric infrastructure (not specifically to NDN).
% \cite{Wahlisch:2012:Backscatter-from}



% % Collaborative detection of DDoS attacks over multiple network domains
% \cite{Chen:2007:Collaborative-detection} distributed change-point detection

% % DDoS flooding attack detection through a step-by-step investigation
% \cite{Jun:2011:DDoS-flooding}

% % A practical method to counteract denial of service attacks
% \cite{Tupakula:2003:A-practical-method}


% Alex: We should not include this reference for the review, but afterwards
% Previous work on DDoS and DoS~\cite{gasti2012ddos}






% Network level (Flooding)
% % 2002
% Pushback~\cite{Pushback} (hop-by-hop push back) aggregate- based congestion control/congestion signatures (e.g., per-destination), rate-limiting, explicit notification
% AITF~\cite{Argyraki:2005:Active-internet} (active filtering requests; path recording, last-point of trust detecting)?
% DefCOM \cite{Oikonomou:2006:A-framework-for-a-collaborative} Communication between src, core, and dst to provide collaborative defense. Dst raises alerts, src differentiate bad/good, packet tagging. Core rate-limit, congestion-aware re-tag. Traffic prioritization.
% StopIt \cite{Liu:2008:To-filter-or-to-authorize:} (filtering) tagging with passports, tagged prioritization, dst- invoked filter instantiation
% NetFence \cite{Liu:2010:NetFence:-preventing} congestion policing: fair queuing (one leaky bucket per {src,L}, L- bottleneck), secure "ticketing" with signals for AIMD to increase/decrease bandwidth allowance
% PSP \cite{Chou:2009:Proactive-surge} topology as origin-destination pairs, packet tagging, OD traffic isolation, sharing based on historic bandwidth demand


% SIFF~\cite{Yaar:2004:SIFF:-A-stateless} (time-based tickets, capabilities)
% TVA \cite{Yang:2005:A-DoS-limiting-network} like SIFF, traffic- based tickets, hierarchical per- AS, per-source fair queuing
% FastPass \cite{Wendlandt:2006:Fastpass:-Providing} (tokens: various puzzles, other methods) traffic authorization, routers verify and filter.
% \cite{Natu:2007:Fine-grained-capabilities} based on TVA/SIFF: specifies previously unspecified mechanisms to grant "tickets" for requesters, extends capabilities to include priorities, changing tickets to be only destination- dependent, not path dependent
% % LazySusan \cite{Crowcroft:2007:Lazy-Susan:} TR (latency-based proof of work--- computational puzzles)
% Portcullis \cite{Portcullis} Protecting connection setup with computational proof of work, tickets and traffic prioritization for the rest, per- computational fairness; DNS to disseminate puzzles
% There are many other puzzle- based schemes, on which Portcullis is based or which problems it solves



% PacketSymmetry \cite{Kreibich:2005:Using-packet} (leverages packet symmetry to detect and remove bad traffic) packet asymmetry - number of transmitted packets (tx) to received packets (rx); negative values when rx outweighs tx, positive values when tx outweighs rx, and zero in the case of perfectly balanced traffic)
% % Detecting SYN flooding attacks
% \cite{Wang:2002:Detecting-SYN-flooding} based on the protocol behavior of TCP SYN-FIN (RST) pairs, and is an instance of the Seqnential Change Point Detection
% dFence \cite{Mahimkar:2007:dFence:-Transparent} (middleboxes) TCP connection proxying, rate estimation and TCP/ACK modification
 

% Overlays:
% CenterTrack \cite{Stone:2000:CenterTrack:-An-IP-overlay} (overlay) reroute suspicious traffic to a special routers with diagnostic capabilities using an overlay
% SOS \cite{Keromytis:2004:SOS:-An-architecture}
% Shield \cite{Kline:2011:Shield:-DoS-filtering}



% NDN and other future information centric architectures were lacking any DDoS mitigation techniques, so we put efforts in understanding DDoS defensive systems, which were made for IP architecture, for the purpose of possible incorporation of them in NDN. In this section we will overview generic directions of DDoS mitigation designs and try to think about which tricks could be used in NDN.  
  
% Capability-based systems allow routers to negotiate, perform and enforce limitations on bandwidth consumption on router-to-router and router-to-client links. Usually such systems require an extensive trust infrastructure in order to validate secure keys used by routers and clients. This implies that each client must pass through an authentication process prior to using any bandwidth~\cite{Capabilities}. We believe that for NDN this method doesn't fit really well for the following reasons. First, this method is very questionable in terms of scalability and mobility support since it requires tight trust connections between routers and hosts. Thus, it neglects one of the main advantages (mobility) of NDN. Second, the design of trust infrastructure for NDN is not well defined yet, which means that we cannot really make assumptions in our design of mitigation techniques.

% Computation-based systems provide each client an access to the network resources only after performing a significant computations such as solving puzzles. Spending a huge amounts of computational resources can effectively slow down a flooding attack by the botnet, however it also creates additional computational burden for legitimate clients. One of the main difficulties in such systems is the question of puzzle dessimination. In ~\cite{Portcullis} authors suggested to propagate puzzles using DNS system. In NDN since routers have caching abilities, we can propagate puzzles just like a normal data packets, using Interest-Data exchange. Overall, this method looks interesting and deserves being investigated in a future work.

% Congestion control systems do per flow traffic analysis and drop of packets belonging to misbehaving flows. For instance, Random Early Detect (RED)~\cite{RED} identifies flows that do not comply with TCP-friendly end-to-end congestion control, and preferentially drop them. Such techniques, if largely deployed, could perform well, however, they cannot provide effective defense against non-greedy botnets that are creating a huge amount of low-bandwidth flows. Since NDN routers maintain a richer per-packet state, which translates into a better per-flow statistics, these techniques can become much more effective when applied to NDN.

% Push back systems are trying to detect bad and good traffic flows on each router, and once attack is detected by one of the routers, it starts a coordinated push back by downrating incoming flows with a bad traffic in order to provide more capacity for a good traffic. This process is reiterating downstream till it reaches edge routers, which are directly connected to attacking bot machines~\cite{Pushback}. In general this technique can be applied to NDN architecture thanks to its decentralized coordination of downrating bad traffic flows. Moreover, NDN provides better capabilities for accurate flow detection.

%were looking for getting accustomed 

%%% Local Variables: 
%%% mode: latex
%%% TeX-master: "paper"
%%% End: 
