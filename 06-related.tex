\section{Related work}
\label{sec:related work}

NDN and other future information centric architectures were lacking any DDoS mitigation techniques, so we put efforts in understanding DDoS defensive systems, which were made for IP architecture, for the purpose of possible incorporation of them in NDN. In this section we will overview generic directions of DDoS mitigation designs and try to think about which tricks could be used in NDN.  
  
% Capability-based systems allow routers to negotiate, perform and enforce limitations on bandwidth consumption on router-to-router and router-to-client links. Usually such systems require an extensive trust infrastructure in order to validate secure keys used by routers and clients. This implies that each client must pass through an authentication process prior to using any bandwidth ~\cite{Capabilities}. We believe that for NDN this method doesn't fit really well for the following reasons. First, this method is very questionable in terms of scalability and mobility support since it requires tight trust connections between routers and hosts. Thus, it neglects one of the main advantages (mobility) of NDN. Second, the design of trust infrastructure for NDN is not well defined yet, which means that we cannot really make assumptions in our design of mitigation techniques.

% Computation-based systems provide each client an access to the network resources only after performing a significant computations such as solving puzzles. Spending a huge amounts of computational resources can effectively slow down a flooding attack by the botnet, however it also creates additional computational burden for legitimate clients. One of the main difficulties in such systems is the question of puzzle dessimination. In ~\cite{Portcullis} authors suggested to propagate puzzles using DNS system. In NDN since routers have caching abilities, we can propagate puzzles just like a normal data packets, using Interest-Data exchange. Overall, this method looks interesting and deserves being investigated in a future work.

Congestion control systems do per flow traffic analysis and drop of packets belonging to misbehaving flows. For instance, Random Early Detect (RED)~\cite{RED} identifies flows that do not comply with TCP-friendly end-to-end congestion control, and preferentially drop them. Such techniques, if largely deployed, could perform well, however, they cannot provide effective defense against non-greedy botnets that are creating a huge amount of low-bandwidth flows. Since NDN routers maintain a richer per-packet state, which translates into a better per-flow statistics, these techniques can become much more effective when applied to NDN.

Push back systems are trying to detect bad and good traffic flows on each router, and once attack is detected by one of the routers, it starts a coordinated push back by downrating incoming flows with a bad traffic in order to provide more capacity for a good traffic. This process is reiterating downstream till it reaches edge routers, which are directly connected to attacking bot machines~\cite{Pushback}. In general this technique can be applied to NDN architecture thanks to its decentralized coordination of downrating bad traffic flows. Moreover, NDN provides better capabilities for accurate flow detection.


%were looking for getting accustomed 

%%% Local Variables: 
%%% mode: latex
%%% TeX-master: "paper"
%%% End: 
